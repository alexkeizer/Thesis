
%LEAN namespace Enhancing
\chapter{Enhancing the QPF formalization}
\label{ch:enhancing}


In the preceding chapter we discussed how to port the formalization of QPFs from Lean3 to Lean4.
In this process, I identified, and ineed implemented various enhancements. 
This chapter serves to elaborate on those changes which are too big to be considered just porting, but don't fall strictly under (co)datatype synthesis and metaprogramming parts of the project.






\section{Functions \& Currying}
Like most functional languages, in Lean it is idiomatic to write functions in their curried form, 
so \leanm{f : Type → Type → Type}, rather than \leanm{f : (Type × Type) → Type}.
However, the formalizations are done in terms of vectors of types and uncurried type functions.

%LEAN namespace Hidden -- hide definition, so it does not interfere
\begin{leancode}
    def TypeFun (n : Nat)
      := TypeVec n → Type v
\end{leancode}
%LEAN end Hidden
Which is a function that takes a list of exactly $n$ types, and returns a type --- the \lean{u} and
\lean{v} refer to the respective universes that these types live in, and betray a technical limitation
of this encoding: all arguments to a \leanm{TypeFun} must live in the same universe.

There is an obvious translation from \leanm{TypeFun} to a curried type function and, vice versa, from a curried function taking $n$ types from the same universe and returning a type, to a \leanm{TypeFun n}.
These conversions were implemented as \leanm{TypeFun.toCurried} and \leanm{TypeFun.ofCurried}, respectively, and it was proven that these functions are isomorphisms. 

To wit, they function as expected:
\begin{leancode}
    variable (F : TypeFun 2) (F' : Type 1 → Type 1 → Type 2)
    example : F.curried α β = F ![α, β]                 := by rfl
    example : F' α β = (TypeFun.ofCurried F') ![α, β]   := by rfl
\end{leancode}

The type \leanm{CurriedFun α β n} is a recursively defined alias for \leanm{α → ... → α → β}, taking $n$ arguments of type \lean{α} to produce an element of \lean{β}.
\begin{leancode}
    abbrev CurriedFun (α : Type u) (β : Type v) : Nat → Type (max u v)
      | 0   => PUnit.{u+1} → β
      | 1   => α → β
      | n+1 => α → CurriedFun α β n
\end{leancode}
Intuitively one might expect a \leanm{CurriedFun} taking no arguments (so, $n = 0$) to be equal to  just \leanm{β}, but that does not typecheck --- \leanm{Type v} and \leanm{Type (max u v)} are not, in general, the same type.
One should be able to lift \lean{β} into the right universe, but such functions are not particularly interesting, so the simpler solution was chosen: functions that take no arguments are functions from the universe polymorphic unit type.

A curried type function is just an instance of \leanm{CurriedFun}.
\begin{leancode}
    abbrev CurriedTypeFun := CurriedFun (Type u) (Type v)
\end{leancode}



Considering all this complexity, it is easy to see why Avigad et al made all the formalizations and
constructions in terms of uncurried functions. 
Still, uncurried functions feel very unidiomatic and users will rightfully expect their (co)datatypes
to function as curried type functions. 
It would be interesting to see whether it is possible to reformulate the formalization of QPFs in
terms of curried functions. 
For the time being, we'll satisfy ourselves with hiding these details throughs
\leanm{TypeFun.curried} and \leanm{TypeFun.ofCurried} conversions.



\section{Make MvQpf automatically assume MvFunctor}
The following change might feel underwhelming, but it presents a considerably quality 
of life improvement for the \leanm{MvQpf} typeclass. The latter was originally defined as
%LEANIGNORE
\begin{leancode}
    class MvQpf {n : Nat} (F : TypeFun n) [MvFunctor F] where
        -- ...
\end{leancode}
This makes sense, \lean{F} can only be a QPF if it is a functor in the first place.
However, when declared like this, the type of \leanm{MvQpf} is 

%LEANIGNORE
\begin{leancode}
    {n : Nat} → (F : TypeFun n) → [MvFunctor F] → Type _
\end{leancode}
In particular, this means we cannot write \lean{MvQpf F}, unless an instance of \lean{MvFunctor F} can be inferred. For concrete QPFs this is generally not problematic, but when \lean{F} is a variable, this restriction becomes annoying. 

For example, if we wish to formalize ``Let \lean{F} be an $n$-ary QPF'', we would like to simply write ``\lean{F} is an $n$-ary type function, and there is an instance of \lean{MvQpf F}''. Like so
\begin{leancode}
    variable (F : TypeFun n) [MvQpf F]
\end{leancode}

This doesn't work, we have to assume a \lean{[MvFunctor F]} bound as well.
Even worse, in some situations different mentions of \lean{MvQpf F} could infer different \lean{MvFunctor F} instances for the implicit argument, causing surprising type mismatches.

So, the definition was changed to 
\begin{leancode}
    class MvQpf {n : Nat} (F : TypeFun n) extends MvFunctor F where
        -- ...
\end{leancode}

Which roughly means that the \lean{MvFunctor F} instance becomes one of the fields of the \lean{MvQpf}
typeclass. Hence, \lean{MvQpf F} has no more implicit arguments (the value for \lean{n} is fixed by \lean{F}), 
which fixes these issues.


\section{Type class inference for vectors}

Composition of an $n$-ary functor \lean{F} with $m$-ary functors \lean{G₀, G₁, ..., Gₙ₋₁} originally took the following variables.

%LEANIGNORE
\begin{leancode}
    variable {n m : Nat} 
             (F : TypeVec.{u} n → Type _) 
             [fF : MvFunctor F] 
             [q : MvQpf F] 
             (G : Fin2 n → TypeVec.{u} m → Type u)
             [fG : ∀ i, MvFunctor (G i)] 
             [q' : ∀ i, MvQpf (G i)]
\end{leancode}

Firstly, by the preceding section, we can leave out the \lean{MvFunctor} assumptions, since they are now part of the \lean{MvQpf} assumptions.

Secondly, the last variable, \lean{q'}, states that \lean{G i}, for every \lean{i} of type \lean{Fin2 n}, is a QPF.
The square brackets indicate that it is a \emph{typeclass variable}, which should be filled in by typeclass \emph{inference}.

As there are only $n$ inhabitants of \lean{Fin2 n}, the universally quantified inference problem \lean{∀ i, MvQpf (G i)} neatly reduces to $n$ non-qualified inference problems \lean{MvQpf (G 0)}, \lean{MvQpf (G 1)}, etc.

However, Lean's inference engine does not seem to be able to make this step by itself, failing to infer an instance for
\lean{∀ i, MvQpf (G i)} even if an instance of \lean{MvQpf} can be inferred for each individual type function \lean{Gᵢ}.

So, we introduce a new typeclass, \lean{VecMvQpf}, which wraps the universally quantified typeclass problem.
\begin{leancode}
    class VecMvQpf (G : Vec (TypeFun m) n) where
        prop : ∀ i, MvQpf (G i)
\end{leancode}

Then we can register instances by recursion on the size $n$ of the vector $G$.
For the base case $n=0$, the vector \lean{G} is empty, and its vacuous to say all elements are QPFs.

%LEANIGNORE
\begin{leancode}
    instance instNil    (G : Vec (TypeFun m) 0) : VecMvQpf G
        := /* ... */
\end{leancode}

For $n+1$, we recurse in the \lean{succ} typeclass variable.

%LEANIGNORE
\begin{leancode}
    instance instSucc   (G : Vec (TypeFun m) (n + 1)) 
                        [zero : MvQpf (G .fz)]
                        [succ : VecMvQpf (fun i => G i.fs)] : 
                            VecMvQpf G 
        := /* ... */
\end{leancode}

There is no need to write typeclass variables in terms of \lean{VecMvQpf} because of the following instance.
%LEANIGNORE
\begin{leancode}
    instance instUnbox [inst : VecMvQpf G] : 
        ∀i, MvQpf (G i)
\end{leancode}

Note that we could write \lean{instNil} and \lean{instSucc} directly in terms \lean{MvQpf}, e.g.,
%LEANIGNORE
\begin{leancode}
    instance instSucc'  (G : Vec (TypeFun m) (n + 1)) 
                        [zero : MvQpf (G .fz)]
                        [succ : ∀ i, G i.fs] : 
                            ∀ i, G i 
        := /* ... */
\end{leancode}
This is accepted, but won't actually help to derive instances for larger vectors \lean{G}.
It seems that Lean puts a limit on recursion depth when trying to infer a universally quantified 
typeclass problem, whereas it will recurse deeper for \lean{VecMvQpf}.





\section{Universe polymorphic finite type}
\label{sec:enhance:pfin2}
Originally, \lean{Fin2} was defined as the following, straightforward, inductive family, as seen in \cref{sec:ind_families}.
\begin{leancode}
    inductive Fin2 : Nat → Type
    | fz : Fin2 (n+1)
    | fs : Fin2 n → Fin2 (n+1)
\end{leancode}

However, this definition forces \lean{Fin2} to live in \Type.
During the project a need arose for a finite type with exactly $n$ inhabitants, but in arbitrary universes. Thus, the \emph{universe polymorphic} \lean{PFin2} type was added.
\begin{leancode}
    inductive PFin2 : Nat → Type u
    | fz : PFin2 (n+1)
    | fs : PFin2 n → Fin2 (n+1)
\end{leancode}
Whence \lean{Fin2} was changed to just be an alias for \lean{PFin2.{0}}.
Of course, most theorems and definitions for \lean{Fin2} were easily redefined in terms of \lean{PFin2}.





\begin{leanhidden}
    end Enhancing

    namespace Procedure
\end{leanhidden}
