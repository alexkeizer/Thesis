
\chapter{Porting the QPF formalization from Lean3 to Lean4}
\label{ch:porting}

Lean is primarily developed by Leonardo de Moura at Microsoft Research. As an in-progress research project, it explicitly provides no stability guarantees, and indeed, Lean 4 is not backwards compatible with Lean 3.
This means that the QPF formalizations by Avigad et al.
% \etal{}
(\cite{avigadDataTypesQuotients2019a}) are not directly usable. The current chapter details my efforts to port these formalizations to Lean 4.


Lean Mathlib is a 





\begin{todo}
    Porting
    \begin{itemize}
        \item Mention Lean3 -> Lean4 differences (instability)
        \item Mention mathlib3 -> mathlib4 (mention that qpfs are part of mathlib3)
        \item Mathport 
                \begin{itemize}
                    \item Takes care of syntactic differences, different naming conventions (mostly)
                    \item Proofs did not translate well (a lot of tactics were missing)
                \end{itemize}
        \item \url{https://leanprover.zulipchat.com/#narrow/stream/287929-mathlib4/topic/mathport.3A.20cases_on.20.3D.3E.20casesOn}
            \begin{itemize}
                \item Implicit \lean{motive} arguments were being inferred differently
                \item \lean{set\_option pp.analyze true}
                \item Fix: be more explicit with arguments and type ascription
            \end{itemize}
    \end{itemize}
\end{todo}