
\chapter{Porting the QPF formalization from Lean 3 to Lean 4}%
\label{ch:porting}

Development of Lean was started by Leonardo de Moura at Microsoft Research~\cite{demouraLeanTheoremProver2015, avigadTheoremProvingLean}. As an in-progress research project, there explicitly are no stability guarantees. Indeed, Lean 4 is not backward compatible with Lean 3, meaning that the QPF formalizations by Avigad \etal{}
(\cite{avigadDataTypesQuotients2019a}) were not directly usable for this project. The current chapter details my efforts to port these formalizations to Lean 4.



Lean 3 has a comprehensive, community-maintained mathematics library, mathlib (\cite{themathlibcommunityLeanMathematicalLibrary2020}), 
which also functions as its unofficial, de-facto standard library, and to which Avigad \etal{} contributed
their formalization of QPFs.
Mathlib4 is the (in-progress) port of this library to Lean 4 (\cite{mathlib4}),
and I am currently in the process of contributing
the ported QPF formalization back to mathlib4.




\section{Mathport \& changes from Lean 3}
Lean 4 is not just a new version of the language; most of the code base has been rewritten.
There were some changes in the syntax and naming convention, but most importantly, 
the meta-programming in Lean 4 has been completely reworked.

To help with porting code, the mathlib community developed the mathport tool (\cite{mathport}), which 
(best-effort) translates Lean 3 source code into Lean 4 source code. This takes care of superficial 
syntax and naming changes, but the translation is far from perfect.
In particular, mathport did reasonably well in translating declaration signatures, 
but the translated proofs were full of errors.
This is primarily because mathport (intentionally) does not deal with meta-programming code at all and a lot
of \emph{tactics} (proof automation meta-programs) provided by mathlib were still missing from mathlib4.


\section{Inference of implicit arguments}
Lean relies heavily on inference to reduce verbosity, falling back on explicitly provided
values or annotations when types or arguments are wrongly inferred.
Lean 4 sometimes inferred something sufficiently different from Lean 3 that the result no longer
typechecked.



For example, in the following theorem, Lean 4 couldn't figure out the right types for \lean{f ::: g} and 
\lean{f' ::: g'}. Explicit type annotations, which look like \lean{(f ::: g : \_)}, had to be added.
\begin{leancode}
  theorem append_fun_inj {α α' : Typevec n} {β β' : Type _} 
                         {f f' : α ⟹ α'} {g g' : β → β'} :
    (f ::: g : (α ::: β) ⟹ _) = (f' ::: g' : (α ::: β) ⟹ _) 
      → f = f' ∧ g = g'
\end{leancode}


Similarly, Lean has a different inference algorithm for so-called \emph{eliminators}.
Lean 3 (and more recent versions of Lean 4) support an \leanm{@[elab_as_eliminator]} attribute on user-defined
functions that instructs Lean to use this specialized eliminator inference, but the version we're using
does not have that.
This came up, e.g., in the proof of \lean{Cofix.bisim\_aux}. The Lean 3 proof simply used
\begin{leancode}    
  apply quot.inductionOn c
\end{leancode}  
Where \lean{quot.inductionOn} is marked with \lean{elab\_as\_eliminator}.
In Lean 4 we had to explicitly provide the \emph{motive} (which determines the induction hypothesis), 
since it was being inferred incorrectly.
\begin{leancode}
  apply Quot.inductionOn (motive := fun c => ∀b, r c b → 
        Quot.lift (Quot.mk r') h₁ c = Quot.lift (Quot.mk r') h₁ b) c
\end{leancode}

The tricky part is that the proof generally doesn't throw an error at \lean{Quot.inductionOn} when it
infers the wrong motive; it just produces a different proof state, tripping up later tactics.
Catching this involved stepping through both the Lean 3 and Lean 4 proof states to find out where, exactly, the proof states diverged, 
and then explicitly providing the desired motive.

By default, the pretty printer used to show intermediate proof states will
omit any implicit arguments in the term, to closely resemble whatever was written in the source code.
However, we can instruct it to be more verbose (both in Lean 3 and in Lean 4):
\begin{leancode}
  set_option pp.implicit true -- print implicits, or
  set_option pp.all true      -- set all options ()
\end{leancode}

This problem was hard to diagnose but relatively easy to remedy.
Once the divergence was found, we can readily see from the Lean 3 proof state what the motive should be.


\section{Missing tactics} 
Conversely, the issue of missing tactics was easy to diagnose, but harder to fix.
The long-term best solution would have been to implement missing tactics.
However, there were quite a few, and implementing them would probably have taken a long time.

Instead, proofs that used such missing tactics were changed to use the lower-level tactics that were
available. For example, the \lean{congr} tactic
was missing, but easily replaced with either \lean{apply congr}, \lean{apply congrArg}, or \lean{apply congrFun}.

Notably, mathlib4 has made much progress in implementing such missing tactics in more recent versions, 
so these proofs can now be changed back to use the higher-level tactics.



