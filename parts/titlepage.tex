
%
%   TITLE PAGE
%
\title{Implementing a definitional (co)datatype package in Lean 4, based on quotients of polynomial functors}
\author{Alex C.\ Keizer}
\birthdate{October 15th, 1998}
\birthplace{Alkmaar, The Netherlands}
\defensedate{T.B.D, 2023}
\supervisor{Dr.\ Jasmin Blanchette}
\supervisor{Dr.\ Benno van den Berg}
\committeemember{Dr.\ Malvin Gattinger (chair)}
\committeemember{Dr.\ Tobias Kappé}
\committeemember{Dr.\ Jasmin Blanchette (supervisor)}
\committeemember{Dr.\ Benno van den Berg (supervisor)}
\maketitle

% Switch to roman numbering for abstract and TOC
\pagenumbering{roman}

\section*{Abstract}
Coinduction is the dual of induction. While inductive types are ubiquitous in functional programming
languages, coinductive types, also known as codatatypes, are considerably less well-supported.
Notably, Lean 4, the latest edition of an interactive theorem prover and dependently typed functional 
programming language developed at Microsoft Research, lacks support for coinduction.

I have implemented a (co)datatype package for Lean 4, which compiles high-level, definitional
specifications for types, which may mix induction, coinduction and quotients, into a definition
of that type in terms of quotients of polynomial functors.
These QPFs were previously formalized in Lean 3, which I took as a starting point and ported to Lean 4.
This thesis describes how the compilation procedure works, and dives into technical details of the Lean
meta-programming system that influenced the implementation.
The package adds \data{} and \codata{} macros for specifications of inductive, respectively coinductive, 
types. Additionally, a \qpf{} macro exposes a specific part of the procedure, called the \emph{composition pipeline},
that is used to define new QPFs composed of other QPFs.





\section*{Acknowledgements}
Naturally, I would like to thank my supervisors, Jasmin Blanchette and Benno van den Berg, for 
introducing me to the world of interactive theorem provers, their guidance, advice and patience.
I would also like to thank Jeremy Avigad and Simon Huddon, for agreeing to meet with me and explain their work, which formed the inspiration of this thesis. Of course, I should also thank Malvin Gattinger and
Tobias Kappé for agreeing to be on the committee. Finally, I want to thank my friends and family, too
numerous to name, for their support and enduring my ramblings about the awesomeness of theorem provers and infinite data structures. Finally, I would like to thank my partner, Alice, in particular, for
being an endless source of love and support when I needed it most.


\tableofcontents


% Switch back to arabic numbering for the rest of the document
\newpage
\pagenumbering{arabic}