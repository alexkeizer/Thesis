
\chapter{Future considerations / current limitations}
\label{ch:limitations}


As mentioned before, the implementation is not yet complete.
This chapter will explain some limitations in what is implemented so far, and discuss
additions that seem useful.

\section{Universe Polymorphism}

Recall from \ref{sec:shape_types} that the procedure starts with synthesizing a ``head'' type that
looks roughly like the following:
\begin{leancode}
  inductive HeadT
    | CtorA : HeadT
    | CtorB : HeadT
\end{leancode}
The names and number of constructors will differ, but crucially Lean will generate a \emph{non} universe 
polymorphic type. That is, \lean{HeadT} lives in \lean{Type 0}, and only \lean{Type 0}.

As a result, any QPF defined with \data{}/\codata{} will \emph{not} be universe polymorphic.
For example, our running example of lists,
\begin{leancode}
  data List α 
    | nil  : List α
    | cons : α → List α → List α
\end{leancode}

The \inductive{} analogue of this definition \emph{is} polymorphic, but because of this limitation,
\lean{List} defined in terms of \data{} is \emph{not}.
  

The solution is fairly simple, we can explicitly state that we want \lean{HeadT} to be polymorphic
--- recall that the child family of types is universe polymorphic because of the use of \lean{PFin2} (see \ref{sec:shape_types, sec:enhance:pfin2})
\begin{badleancode}

  inductive HeadT : Type u
    | CtorA : HeadT
    | CtorB : HeadT

\end{badleancode}

However, there is a bug in the version of Lean we're using (\texttt{nightly-2022-04-28}) 
which causes it to reject the above. 
This bug is already fixed in more recent versions of Lean, 
so this limitation should be relatively easy to resolve.



\section{Constructors and Characteristic Theorems}

Currently, a user has to employ \lean{MvQpf.Fix.mk} or \lean{MvQpf.Cofix.mk} to construct elements
of data-, resp. codatatypes. For example, the constructors for \lean{List} can be defined as:
\begin{leancode}
  def List.nil := MvQpf.Fix.mk MvQpf.Fix.mk List.Shape.nil
  def List.cons := 
    fun a as => MvQpf.Fix.mk MvQpf.Fix.mk (List.Shape.cons a as)
\end{leancode}
These should just be automatically generated.


Similarly, (co)recursion principles are available through \lean{MvQpf.Fix.drec} (dependent recursor) 
and \lean{MvQpf.Cofix.corec}, but because those are generic their signature is quite obscure.
\begin{leancode}
  #check @MvQpf.Fix.drec _ List.Internal _
  -- {α : TypeVec 0} →
  --   {β : MvQpf.Fix List.Internal α → Type} →
  --     ((x : List.Internal (α :::ᵥ Sigma β)) → 
  --        β (MvQpf.Fix.mk ((TypeVec.id ::: Sigma.fst) <$$> x))
  --     ) →
  --       (x : MvQpf.Fix List.Internal α) → β x
\end{leancode}

The \inductive{} version of \lean{List} defines its recursion principle as
\begin{leancode}
  def List.rec {α : Type _} {motive : QpfList α → Sort _} :
  motive QpfList.nil 
    → ((head : α) → (tail : QpfList α) → motive tail → motive (QpfList.cons head tail))
    → (t : QpfList α) 
    → motive t
\end{leancode}

Which is a lot easier to read.
We can define the second recursion principle fairly easily in terms of the generic \lean{MvQpf.Fix.drec}.
\begin{leancode}
  def List.rec {α : Type _} {motive : QpfList α → Sort _} : /* ... */ :=
    fun nil cons => MvQpf.Fix.drec (fun x => 
      match x with
      | .nil            => nil
      | .cons head tail => cons head tail.fst tail.snd
    )
\end{leancode}

We should probably automatically generate such specialized (co)recursion principles for all (co)datatypes.
Similarly, \lean{MvQpf.Fix.ind} gives a generic induction principle, but it would be nicer to generate
specialized induction principles.


The last characteristic theorem is that of no confusion; i.e., that different combinations of constructors
produce different elements. This principle is less straightforward, there does not exist a generalized
version of it yet, so more work is needed to see how no confusion theorems would be generated.



\section{User-friendly (co)recursion and (co)induction}
If we just want do structural recursion on a single variable, it's relatively easy to do so directly
in terms of the recursion principle.
For example, recursively determining the length of a list.
\begin{leancode}
def List.length : List α → Nat :=
  MvQpf.Fix.rec fun as => match as with
    | .nil                => 0
    | .cons a (as : Nat)  => as + 1 
\end{leancode}

It is important that Lean can statically determine that recursive functions terminate for all 
arguments. \footnote{Lean \emph{can} admit non-terminating functions when they're marked partial, but partial functions are not allowed to be used in proofs}
In the case of structural recursion on a single variable, such as the example above, termination is straightforward,
but Lean also has a sophisticated \emph{equation compiler} that makes it easy to define functions that recurse
on multiple variables, with support for termination hints for when it fails to automatically proof
termination.

It would be interesting to see if the existing equation compiler mechanisms can be re-used
to also provide a high-level way of defining (co)recursive functions for QPFs.

% \subsection*{Coinduction}

% Analogous to (co)recursive functions are proofs using (co)induction.

% \begin{todo}
%     \begin{itemize}        
%         \item User friendly (co)recursion / (co)induction
%                 \begin{itemize}
%                     \item `eliminator`
%                 \end{itemize}
%     \end{itemize}
% \end{todo}



