
%
%   PACKAGES   
%

\usepackage{url}
\usepackage{hyperref}
\usepackage{illcmolthesis}
\usepackage{amsmath}
\usepackage{amssymb}
\usepackage{unicode-math}
\usepackage{verbatim}   % for the comment environment
\usepackage{cleveref}
\usepackage{xcolor}
\usepackage{framed}
\usepackage{booktabs}
\usepackage{enumitem}

\usepackage{tikz}
\usepackage{tikz-cd}

% 
%   Minted
% 

\usepackage[outputdir=./out]{minted}

% instruct minted to use our local theorem.py
\newmintinline[leanm]{lean4.py:Lean4Lexer -x}{fontsize=\small}
\newminted[leancode]{lean4.py:Lean4Lexer -x}{fontsize=\small}

% Lean code that does not typecheck/compile
\newminted[badleancode]{lean4.py:Lean4Lexer -x}{%
    fontsize=\small,%
    frame=leftline,%
    framesep=0mm,%
    rulecolor=red%
}

% Disable the red boxed minted will put around "syntax errors" 
\AtBeginEnvironment{minted}{%
  \renewcommand{\fcolorbox}[4][]{#4}}




%   
%   Fonts & coloring
% 

\usepackage[utf8]{inputenc}
\usepackage{newunicodechar}
\usepackage{fontspec}



% switch to a monospace font supporting more Unicode characters
\setmonofont{JetBrains Mono NL Light}
% \setmonofont{Droid Sans Mono}

\newfontfamily{\freemono}{FreeMono}
\newfontfamily{\droidmono}{Droid Sans Mono}
\newfontfamily{\jbmono}{JetBrains Mono NL Light}


% Colors
\definecolor{keywordcolor}{HTML}{008000}
\definecolor{operatorcolor}{HTML}{008000}

% Unicode glyphs

\newcommand\iflean[2]{%
\ifx\leanmode\undefined%
#2%
\else%
#1%
\fi%
}

\newcommand\leanoperator[1]{%
\ifx\leanmode\undefined%
#1%
\else%
{\color{operatorcolor} #1}%
\fi%
}

% TODO: make it so that these don't mess up monospacing in code environment
\newcommand\leanmathoperator[1]{\ensuremath{\leanoperator{#1}}} 
\newunicodechar{→}{\leanmathoperator{\rightarrow}}
\newunicodechar{⟹}{\leanmathoperator{\Longrightarrow}}
\newunicodechar{⋅}{\leanmathoperator{\cdot}}

\newunicodechar{α}{{\droidmono α}}
\newunicodechar{β}{{\droidmono β}}
\newunicodechar{γ}{{\droidmono γ}}
\newunicodechar{ρ}{{\droidmono ρ}}
\newunicodechar{₁}{\ensuremath{_\text{1}}}
\newunicodechar{₂}{\ensuremath{_\text{2}}}
\newunicodechar{ᵢ}{\ensuremath{_\text{i}}}
\newunicodechar{ⱼ}{\ensuremath{_\text{j}}}
\newunicodechar{ₖ}{\ensuremath{_\text{k}}}
\newunicodechar{ₙ}{\ensuremath{_\text{n}}}
\newunicodechar{ₘ}{\ensuremath{_\text{m}}}
\newunicodechar{₋}{\ensuremath{_\text{-}}}

% 
% \usepackage[GreekAndCoptic]{ucharclasses}
% \setTransitions{GreekAndCoptic}{}{}


% 
%   Spacing & Layout
% 

\usepackage{geometry}

\geometry{
    hmargin=10em
}
\frenchspacing
\setlength{\parskip}{6pt}
\setlength{\parindent}{0pt}



% 
%   Custom environments
% 
\definecolor{shadecolor}{HTML}{F8E0E0}

% definitions
\newenvironment{definition}[1][Definition:]{\begin{trivlist}                         
    \item[\hskip \labelsep {\bfseries #1}]}{\end{trivlist}}
    
% remark    
% \newenvironment{remark}{\begin{trivlist}                         
%   \item[\hskip \labelsep {\bfseries Remark:}]}{\end{trivlist}}



\newenvironment{remark}{%
% \def\FrameCommand{\colorbox{remarkcolor}}%
% \MakeFramed{\advance\hsize-\width \FrameRestore}%
\begin{framed}
\begin{trivlist}
    \item[\hskip \labelsep {\bfseries Remark:}]}%
{%
\end{trivlist}%
\end{framed}
    % \endMakeFramed%
}
    

% todo
\newenvironment{todo}{%
\definecolor{shadecolor}{HTML}{F8E0E0}%
\begin{shaded}%
\begin{trivlist}                         
    \item[\hskip \labelsep {\bfseries Todo:}]}{\end{trivlist}\end{shaded}}

% hidden code    
\newenvironment{leanhidden}{\expandafter\comment}{\expandafter\endcomment}



% Add a new list, whose references are called "step 1", "step 2", etc.
% \newlist{stepenum}{enumerate}{10}
% \setlist[stepenum,1]{label*=\arabic*., ref=(\arabic*)}
% \setlist[stepenum,2]{label*=(\alph*)}

% \crefname{stepenumi}{step}{stepenum}
% \Crefname{stepenumi}{Step}{stepenum}


% 
%   Custom commands
% 

\newcommand\lean[1]{%
\ifx\leanmode\undefined%
\def\leanmode{1}%
\texttt{\small #1}%
\undef\leanmode%%
\else%
\texttt{#1}%
\fi%
}

\newcommand\keyword[1]{{\color{keywordcolor} \textbf{\lean{#1}}}}

\newcommand\inductive{{\keyword{inductive}}}
\newcommand\data{\keyword{data}}
\newcommand\codata{\keyword{codata}}
\newcommand\qpf{\keyword{qpf}}
\newcommand\ldef{\keyword{def}}

\newcommand\Type{\leanm{Type}}
\newcommand\Typen[1]{\leanm{Type #1}}


\newcommand\etal{\emph{et al.}}