\chapter{Background}%
\label{ch:background}

A key concept for this thesis is the QPF (Quotient of Polynomial Functor), and how QPFs can be used to encode (co)inductive types. The current chapter serves to illuminate this notion, and explain relevant parts of the Lean system along the way.

We will assume as little background knowledge as possible, yet, some (minimal) exposure to category theory and functional programming concepts will be beneficial to understanding. Readers can find references at~\cite{awodeyCategoryTheory2010,milewskiCategoryTheoryProgrammers2019} for category theory and~\cite{christiansenFunctionalProgrammingLean} for functional programming.

% \begin{todo}
%     Add references for FP, maybe The Little Typer?
% \end{todo}

\begin{remark}
    We will use remarks like this one below code snippets to explain Lean syntax and concepts that might not be familiar. We don't assume any knowledge of Lean, but the interested reader is invited to consult the online documentation, or \emph{Functional Programming in Lean} for a more comprehensive introduction~\cite{avigadTheoremProvingLean,christiansenFunctionalProgrammingLean}.
\end{remark}

In the interest of consistency, we will use Lean 4 syntax and naming conventions throughout the whole thesis,
even in our current discussion of the general theory and formalizations as presented by Avigad \etal{} (in Lean 3).
The differences between the Lean definitions presented here and their original definition in~\cite{avigadDataTypesQuotients2019} are purely superficial.



The encoding of types as QPFs relies on a key observation; a lot of types are functorial in nature.
Take, for example, the inductive type of lists, specified as
\begin{leancode}
  inductive List (α : Type)
    | nil  : List α
    | cons : α → List α → List α
\end{leancode}

This defines a function \lean{List} that takes a type, α, and returns a new type, whose elements represent lists of α.
We call such functions \emph{type functions}.

\section{Type universes}
Lean is a dependently typed language, so types \emph{are} terms. 
In particular, things like \lean{Nat} and \lean{String} are types, but they are also \emph{values} of type \Type{}.
The type of \lean{List}, then, is \leanm{Type → Type}.
In particular, there is no distinction between type functions and functions that operate on, e.g., natural numbers.

% Since types are first-class terms in our language,  no different from how the signature of, e.g., \lean{Nat.add} (binary addition of natural numbers) is \leanm{Nat → Nat → Nat}.

\begin{remark}
    We write \leanm{f : α → β} to say ``\lean{f} has type \leanm{α → β}''.
    The arrow type \lean{α → β} stands for the type of functions taking an argument of type \lean{α}
    to produce a value of type \lean{β}.
\end{remark}

We call \Type{} a \emph{type universe}, i.e., a type whose elements are themselves types. However, \Type{} is itself also a value, which needs to live in some universe.
It cannot be the case that \leanm{Type : Type}; this leads to Girard's paradox, which is the type theory analogue of Russel's paradox~\cite{girardInterpretationFonctionelleElimination1972}.

Instead, Lean has an infinite sequence of increasing type universes, so that \leanm{Type : Type 1}, \leanm{Type 1 : Type 2}, and in general, \leanm{Type u : Type (u+1)}. In fact, \leanm{Type} is just a shorthand for \leanm{Type 0}, the smallest type universe.

Universe levels \lean{u} are essentially natural numbers, but they are \textbf{not} first-class values, and are indeed very different from elements of \leanm{Nat}.
In particular, when writing \Typen{u}, the universe level \lean{u} is \textbf{not} a regular term. There is a separate grammar, with some minimal builtin operations, that defines valid universe levels.

It is possible, and indeed common to be generic over the universe of some type parameter, we call such 
definitions \emph{universe polymorphic}.

For example, the actual definition of lists specifies that the parameter \lean{α} should live in an arbitrary universe \lean{Type u}.
\begin{leancode}
  inductive List (α : Type u)
    | nil  : List α
    | cons : α → List α → List α
\end{leancode}

The type of this version of \lean{List} is \leanm{Type u → Type u}, for arbitrary universe levels \lean{u}.

\section{Functoriality of type functions}

Returning to our discussion of \lean{List} and its functoriality, there is an obvious mapping function to lift a function \lean{f : α → β}, for arbitrary types \lean{α} and \lean{β}, into a function \lean{List α → List β}, by applying \lean{f} to each element of the argument list. Its signature is written as
\begin{center}
    \leanm{map : (f : α → β) → List α → List β}
\end{center}

\begin{remark}
    If a function takes multiple arguments, it is idiomatic to write them in a \emph{curried} style,
    so \leanm{f : α → β → γ} says that \lean{f} is a function that takes two arguments, an \lean{α} and a \lean{β}, to produce a \lean{γ}. Arrows are right-associative.

    Furthermore, \leanm{(a : α) → β} is a \emph{dependent} arrow; it is a function from \lean{α} to \lean{β}, with the possibility to make the resulting type depend on the \emph{value} \lean{a} of the first argument. 
    We used this syntax in the definition of \lean{map} above, even though it is a regular, non-dependent function, just to give a name to the first argument. 
    It would have been equivalent to write \leanm{map : (α → β) → List α → List β}.

    Notice that \lean{α} and \lean{β} are arbitrary and thus must also be arguments to \lean{map}. 
    Nevertheless, the value of \lean{α} and \lean{β} can be inferred from the other arguments and there is no need to supply values for them when calling \lean{map}, hence, they can be \emph{implicit arguments}. 
    
    It is possible to explicitly define implicit arguments, using curly brackets \lean{\{...\}}.
    \begin{center}
        \leanm{map: {α β : Type u} → (α → β) → List α → List β}
    \end{center}
    This allows us to call \lean{map} as \lean{map f as} and the values of \lean{α} and \lean{β} are inferred
    from the types of \lean{f} and \lean{as}.

    Lean will automatically add implicit binders for free variables in a type signature,
    a feature called \emph{auto-bound implicits}. 
    We will generally rely on this feature, adopting the convention that \lean{α} and \lean{β} refer to types.
\end{remark}

A type function \leanm{F : Type u → Type v} together with a mapping operation \lean{map : (f: α → β) → F α → F β} form a \emph{functor}, so long as they preserve: 
\begin{itemize}
    \item \emph{Identity maps}, that is, \lean{F (id α) = id (F α)}, with \lean{id β} the identity function on arbitrary types \lean{β}, and
    \item \emph{Compositions}, that is, \lean{(map f) ∘ (map g) = map (f ∘ g)}, where \lean{∘} denotes function composition
\end{itemize}

% Other common examples of functors are products (\lean{F α = α × β}) and sums (F α = \lean{α ⊕ β}), for
% some fixed type \lean{β}.

A function $f : F(α) → α$, for some fixed \emph{carrier} type \lean{α}, is called an \emph{\lean{F}-algebra}.
Inductive types correspond to the carrier of an \emph{initial} such algebra. To clarify, an \lean{F}-algebra \lean{f} is initial, if for every \lean{F}-algebra $g : F(β) → β$ there is exactly one arrow $\lean{rec} : α → β$ that makes the left square of \cref{fig:initial_alg_square} commute.
\begin{figure}[ht]
    \begin{center}
% https://tikzcd.yichuanshen.de/#N4Igdg9gJgpgziAXAbVABwnAlgFyxMJZABgBpiBdUkANwEMAbAVxiRADEAKAHW8bQAWdAJQgAvqXSZc+QijIAmKrUYs2vfkPGSQGbHgJEF5ZfWatEHHtwBGMHCO1T9so6SXUzay7zsPxyjBQAObwRKAAZgBOEAC2SGQgOBBIAIzUDHR2DAAK0gZyIFFYwQI4IJ6qFiARTjUx8YiJyUjGKuZsXLw4MAAeOAPAUTAAxmKiEpENrdQtiADMlR2WwRUgmdl5LoaWDDAR5ZP1cWmzKQtL3iDdfQM4Q6NiAWJAA
   % https://tikzcd.yichuanshen.de/#N4Igdg9gJgpgziAXAbVABwnAlgFyxMJZABgBpiBdUkANwEMAbAVxiRADEAKAHW8bQAWdAJQgAvqXSZc+QijIAmKrUYs2vfkPGSQGbHgJEF5ZfWatEHHtwBGMHCO1T9so6SXUzay7zsOnutIGcsgArCaeqhZWGgyCjhLOMoYo4R4q5up8cVqJgS4pyADsERneMbb2CTp6ySEl6V7RvlXiyjBQAObwRKAAZgBOEAC2SGQgOBBIAIzUDHR2DAAKQa6WA1idAjggkZmWfQGDI2PUk0jGZdFcvDgwAB44T8ADMADGYqJ5x6OIl+eIADMe3KnV2IHmixWBTkEJgfR23yGv1mEymQJBzW4d0ez1eHyOyKQ4TRSAALJi2G9wZCYMtVik4QjCSdEAA2M7oilXNg3bEPJ44YBvCD4z4s34lUnsymWKASpBSgEkppZHGC4Wi95iNpiIA
\begin{tikzcd}
    F(\alpha) \arrow[dd, "f"'] \arrow[rr, "F(\texttt{rec})"] &  & F(\beta) \arrow[dd, "g"'] &  &  & F(\alpha)              &  & F(\beta) \arrow[ll, "F(\texttt{corec})"']           \\
                                                             &  &                          &  &  &                        &  &                                                    \\
    \alpha \arrow[rr, "\texttt{rec}"]                        &  & \beta                    &  &  & \alpha \arrow[uu, "c"] &  & \beta \arrow[uu, "d"] \arrow[ll, "\texttt{corec}"']
    \end{tikzcd}       
\end{center}

\caption{Commuting square for initial $F$-algebras $f$ and final $F$-coalgebras $c$}%
\label{fig:initial_alg_square}
\end{figure}


Consider the type of natural numbers, \lean{Nat}; we'll show that the constructors for \lean{Nat} 
constitute an initial algebra for functor the ${F_{nat}(α) = \lean{Unit} \mathop{⊕} α}$, where \lean{Unit} is the unit type with \lean{(unit : Unit)} as its sole inhabitant and \lean{⊕} denotes a sum. On functions $f : α → β$, the obvious choice of mapping operation $F(f) : F(α) → F(β)$ is such that $F(f)(\lean{unit}) = \lean{unit}$ and $F(f)(a) = f(a)$ for all $a : α$.

Natural numbers are defined by a constant $0 : \lean{Nat}$ and a function $\mathrm{succ} : \lean{Nat} → \lean{Nat}$.
Let $f : F_{nat}(\lean{Nat}) → \lean{Nat}$ be the $F$-algebra defined as $f(\lean{unit}) = 0$ and $f(i) = \mathrm{succ}(i)$ for all $i : \lean{Nat}$.

Then, let $g : F_{nat}(α) → α$ be an arbitrary $F$-algebra, and define a function $\mathrm{rec} : \mathbb{N} → α$ such that $\mathrm{rec}(0) = g(\lean{unit})$ and  $\mathrm{rec}(n+1) = \mathrm{rec}(g(n))$, for every $n ∈ \mathbb{N}$.
The inductive properties of \lean{Nat} are exactly what makes this algebra initial.


Conversely, initial algebras are unique up to isomorphism, so any other initial $F_{{nat}}$-algebra $h: F_{{nat}}(β) → β$
is equivalent to $f$ and its carrier, $β$, is equivalent to \lean{Nat}.


Dually, there is a connection between coinductive types and \emph{final coalgebras}.
More concretely, a $F$-coalgebra is a function $c : α → F(α)$, with $α$ again an arbitrary carrier type,
The coalgebra $c$ is \emph{final} if for every other $F$-coalgebra $d : β → F(β)$ there is a unique arrow
$\lean{corec: β → α}$ that makes the right square of \cref{fig:initial_alg_square} commute.
Finally, the carrier of a final coalgebra corresponds to a coinductive type.


Alternatively, we know that $α$ is the carrier of an initial $F$-algebra (resp.\ final $F$-coalgebra) and
thus corresponds to an inductive (resp.\ coinductive) type iff it is the least (resp.\ greatest) 
fixpoint of $F$. We will also call these the \emph{fixpoint}, resp.\ \emph{cofixpoint}, of $F$.



Not all functors, though, have initial algebras, or final coalgebras.


\section{Polynomial functors}
Of special interest are \emph{polynomial functors}, which, intuitively, are created from just a few primitive operations (constants, sums, products, and exponentials).
More formally, we say that a polynomial functor is defined by a set $A$ and an $A$-indexed family of sets $B_a$ as
\[
    P(X) = \Sigma_{a ∈ A} B_a \rightarrow X  
\]
That is, $P(X)$ is the disjoint union of all functions from $B_a$ to $X$, for every $a ∈ A$.
In the theory, we generally still refer to functors that are not defined in this form, but are isomorphic to a polynomial functor in the strict sense, as polynomial.

To encode this in Lean, we replace ``set'' with ``type'' and obtain the following.
\begin{leancode}
    structure PFunctor := (A : Type u) (B : A → Type u)
\end{leancode}

\begin{remark}
    \leanm{structure} is a simple wrapper around \inductive{}, for when there is only one constructor.
    The above type is equivalent to
    \begin{leancode}
        inductive PFunctor 
        | mk : (A : Type u) → (B : A → Type u) → PFunctor
    \end{leancode}
\end{remark}

Then, the operations on types and functions are straightforward:
\begin{leancode}
    /-- Applying `P` to an object of `Type` -/
    def PFunctor.Obj (P : PFunctor) (α : Type u)
        := Σ x : P.A, P.B x → α

    /-- Applying `P` to a morphism of `Type` -/
    def PFunctor.map (P : PFunctor) (f : α → β) : P.Obj α → P.Obj β 
        := fun ⟨a, g⟩ => ⟨a, f ∘ g⟩
\end{leancode}

\begin{remark}
    Variable \lean{P} is known to be of type \lean{PFunctor}, so \lean{P.Obj} is recognized as \lean{PFunctor.Obj P}. Similarly, as \lean{PFunctor} has only one constructor, \lean{PFunctor.mk}, the anonymous constructor syntax \lean{⟨a, g⟩} is translated to \lean{PFunctor.mk a g}.
    Finally, \leanm{fun ⟨a, g⟩ => _} defines a function that takes a single argument \lean{x : P.Obj α}, and immediately deconstructs it into the constituent elements \lean{a : P.A} and \lean{g : P.B a → α}.
\end{remark}

So an element of \lean{P.Obj α} is a (dependent) pair of a \emph{shape} $a ∈ A$ and a function $g : B_a \rightarrow \alpha$, representing the \emph{contents}. A mapped function \lean{P.map f} then leaves the shape as is, and pre-composes $f$ with the content $g$.

\subsection*{W-types}
We already saw that inductive types are freely generated by the constructors. In a sense, an element of, e.g., \lean{Nat} is a well-founded (i.e., finite) tree with two kinds of nodes: \lean{zero} nodes are leaves, and \lean{succ} nodes have exactly one child.

The \emph{W-type} of a polynomial functor \lean{P} is the type of exactly such trees: shapes $a ∈ A$ distinguish different kinds of nodes, and the cardinality of $B_a$ determines the number of children nodes of type $a$ have. Such trees are easily encoded by an \inductive{} type.

\begin{leancode}
    inductive W (P : PFunctor)
    | mk (a : P.A) (f : P.B a → W P) : W P
\end{leancode}

By construction, the W-type of $P$ is its fixpoint, with \lean{W.mk} its initial algebra.

It is important to make a distinction between types that \emph{are} polynomial functors, and types that are (equivalent to) W-types \emph{of} polynomial functors. 
For example, \lean{Nat} is not a polynomial functor, but it \emph{is} the W-type of a polynomial functor (namely, \lean{F$_\lean{nat}$}). 

Conversely, \lean{List} is both. It is a polynomial functor: take $A = \mathbb{N}$ and $B_n = \{0,..,n-1\}$, then a list of $n$ elements is encoded as a pair with shape $n$, and content $f : \{0,..,n-i\}$ mapping each $i < n$ to the $i$-th element of the list. 
Simultaneously, \lean{List α}, for every \lean{α} is also the W-type of a \emph{different} polynomial functor, defined by $A = \texttt{Unit} ⊕ α$, where $B$ of the unit value is the empty type and $B_a = \lean{Unit}$ for every $a : α$. The W-type of this polynomial functor consists of trees where leaves represent the empty list, and internal nodes are labelled with an element of type \lean{α} (the head of the list) and have exactly one subtree (the tail of the list).

\subsection*{M-types}
Coinductive types are similar, except that the trees they are represented by may be of infinite depth. Encoding these in Lean is quite a bit more involved.

An \emph{M-type} of a polynomial functor \lean{P} is the type of potentially infinite depth trees, where shapes $a ∈ A$ distinguish different kinds of nodes, and the cardinality of $B_a$ determines the number of children nodes of type $a$ have.

An \emph{approximation} of an M-type up to depth $n$ is the type of such trees of height at most $n$, where any required subtrees at depth $n+1$ are replaced with a special ``continue'' leaf. 

\begin{leancode}
  /-- `CofixA P n` is an `n` level approximation of an M-type -/
  inductive CofixA (P : PFunctor) : Nat → Type u
  | continu : CofixA 0
  | intro {n} : (a : P.A) → (P.B a → CofixA P n) → CofixA P (succ n)
\end{leancode}

Then, we define what it means for approximations \emph{agree} --- either the first approximation must be a ``continue'' leaf or both approximations have nodes of the same kind as root, and the corresponding subtrees recursively agree --- whence an M-type is just an infinite series of approximations of increasing depth where every approximation agrees with the next.

\begin{leancode}
  inductive Agree : ∀ {n : Nat}, CofixA P n → CofixA P (n + 1) → Prop
  | continu (x : CofixA P 0) (y : CofixA P 1) : Agree x y
  | intro {n} {a}   (x  : F.B a → CofixA P n) 
                    (x' : F.B a → CofixA P (n + 1)) 
                    : (∀ i : F.B a, Agree (x i) (x' i))
                    → Agree (CofixA.intro a x) (CofixA.intro a x')

  structure M (P : PFunctor) := 
    (approx : ∀ n, CofixA P n)
    (consistent : ∀ n, Agree (approx n) (approx (n+1)))
\end{leancode}

The M-type of a polynomial functor is, by construction, its cofixpoint.

\section{Quotients}
As mentioned in the introduction, Lean also supports quotients of types.
Recall the example of multisets, defined as a quotient on lists.

%LEANIGNORE
\begin{leancode}
    /-- `List.perm as bs` holds iff `as` is a permutation of `bs` -/
    def List.perm : List α → List α → Prop

    def Multiset α := Quot (@List.perm α)
\end{leancode}

Polynomial functors cannot represent such quotients. Thus, we generalize to \emph{quotients} of polynomial functors, QPFs.

Intuitively, a functor $F$ is the quotient of some polynomial functor $P$ when there is a surjective \emph{natural transformation} \lean{abs} from $P$ to $F$. We can think of \lean{abs$_{α}$} as mapping abstract objects in $F(α)$ to their concrete representations in $P(α)$, for every $α$.
Recall that every functor is equipped with a mapping operation on functions $f : α → β$. To say that \lean{abs} is a natural transformation is to say that it respects this functorial structure. That is to say, the square of \cref{fig:nat_transform_square} should commute for all arrows $f : α → β$.
\begin{figure}[htbp]
    \begin{center}
        % https://tikzcd.yichuanshen.de/#N4Igdg9gJgpgziAXAbVABwnAlgFyxMJZABgBpiBdUkANwEMAbAVxiRAAUAKAHW8bQAWdAJQgAvqXSZc+QijIAmKrUYs2AMR58GgkeMkgM2PASILyy+s1aIOWgEYwceiVOOyzpJdStrbm3kdnUVdDaRM5ZAAWCx9VGxBefiF9NxlTFAA2WJVrNkCnOnFlGCgAc3giUAAzACcIAFskMhAcCCQARji82zp7OAB9JJ0U0LrG5uo2pHNcvztqkINxpsRZ6cQAZm75vsGC51SQFc6p9q2dhM1Fo5PEGNbzgFZLtmrisSA
\begin{tikzcd}
    P(\alpha) \arrow[dd, "abs_\alpha"] \arrow[rr, "P(f)"] &  & P(\beta) \arrow[dd, "abs_\beta"] &  & \alpha \arrow[rr, "f"] &  & \beta \\
                                                          &  &                                  &  &                        &  &       \\
    F(\alpha) \arrow[rr, "F(f)"]                          &  & F(\beta)                         &  &                        &  &      
    \end{tikzcd}
    \end{center}

    \caption{Commuting square for natural transformation $abs : P ⟹ F$}%
    \label{fig:nat_transform_square}
\end{figure}


Formally, we can show that \lean{abs} is surjective by providing, for every $α$, a function $\lean{repr}: F(α) → P(α)$ and proving that it is a right-inverse.
This is specified in Lean with the \lean{QPF} typeclass.
\begin{leancode}
  class Qpf (F : Type u → Type u) [Functor F] where
    P        : PFunctor
    abs      : ∀ {α}, P.Obj α → F α
    repr     : ∀ {α}, F α → P.Obj α
    abs_repr : ∀ {α} (x : F α), abs (repr x) = x
    abs_map  : ∀ {α β} (f : α → β) (p : P.Obj α), 
                    abs (f <$> p) = f <$> abs p
\end{leancode}
Where \lean{f <\$> x} is Lean syntax for applying \lean{map$_\texttt{G}$ f} to \lean{x} (the functor \lean{G} is inferred from the type of \lean{x}), \lean{abs\_repr} shows that \lean{repr} is a right-inverse of \lean{abs}, and
\lean{abs\_map} is the naturality condition of \lean{abs}.



\subsection*{Fixpoint and cofixpoint}
Crucially, every QPF still has an initial algebra; if $F$ is the quotient of some polynomial functor
$P$, then the least fixed point of $F$ can be constructed as a quotient of $P$'s W-type over a suitable
relation.
Similarly, the greatest fixed point (thus, the final coalgebra) of a QPF $F$ can be constructed as
a quotient of $P$'s M-type.

The exact details of this construction are not relevant for the rest of this thesis, we merely need to know that there is some way to construct these fixed points. Nonetheless, the full construction can be found in the work of Avigad \etal{}.~\cite{avigadDataTypesQuotients2019}.







\section{Multivariate functors}%
\label{sec:mvfunctor}

Other common examples of functors are sums (\lean{α ⊕ β}) and products (\lean{α × β}).
These functors take two arguments (both \lean{α} and \lean{β}), yet, we've only discussed univariate functors (i.e., functors with a single argument).

One could see, e.g., the product functor as one of two univariate functors, \lean{(α × $\cdot$)} or \lean{(⋅ × β)}, effectively choosing to fix one of the arguments. However, this limits compositionality. To illustrate, consider a type of binary trees.
\begin{leancode}
  data BinaryTree α
    | node : (BinaryTree α × BinaryTree α) → TreeOfProd α
    | leaf : α → TreeOfProd α
\end{leancode}
    Because \lean{BinaryTree α} is used as \emph{both} arguments to the product, this specification is only well-formed if the product functor is functorial in both arguments (simultaneously).
    Thus, we generalize the definitions presented so far to \emph{multivariate} functors. 


It is difficult to reason about $n$-ary curried functions, where $n$ is arbitrary, so instead an \emph{uncurried} representation of multivariate, universe polymorphic, type functions is used.

\begin{leancode}
    def TypeFun (n : Nat) := (TypeVec.{u} n) → Type v
\end{leancode}
Conceptually, \lean{TypeVec} (short for \emph{type vector}) is a list of exactly $n$ elements of \Typen{u}.
We define type vectors as functions from a canonical finite type \lean{Fin2 n}, which has exactly $n$ inhabitants,
to types.
\begin{leancode}
    def TypeVec (n : Nat) := (i : Fin2 n) → Type u
\end{leancode}
Notice that all arguments to a \lean{TypeFun} live in the same universe $u$, but the result lives in a potentially different universe $v$. 
Most constructions require that these universes coincide, which can be written as \lean{TypeFun.\{u,u\}}.

Suppose that \lean{Sum'} is the uncurried version of sums, then \lean{Sum' ![α₁, α₂]} would mean \lean{α₁ ⊕ α₂}.
A multivariate \lean{map} operation now takes not one function, but a vector of $n$ functions, each going from \lean{αᵢ} to \lean{βᵢ}. We define the type of such vectors of functions as,
\begin{leancode}    
    def Arrow (v₁ v₂ : TypeVec n) := (i : Fin2 n) → (v₁ i → v₂ i)
\end{leancode}  
We generally write \lean{v₁ ⟹ v₂} instead of \lean{Arrow v₁ v₂}, leading to the following signature
for multivariate \lean{map}.

\begin{leancode}
    map : {v₁ v₂ : TypeVec n} → (v₁ ⟹ v₂) → F v₁ → F v₂
\end{leancode}
\begin{remark}
    Objects of type \lean{TypeVec n}, for some fixed but arbitrary length $n$, and arrows \leanm{(⋅ ⟹ ⋅)} form a category. 
    Multivariate type functors are in fact nothing more than (univariate) functors from the category of type vectors to the category of types.
\end{remark}

Thus, if \lean{f₁ : α₁ → β₁} and \lean{f₂ : α₂ → β₂}, then \lean{Sum'.map ![f₁, f₂]} yields a function from \lean{Sum' ![α₁, α₂]} to \lean{Sum' ![β₁, β₂]}.

Finally, the functoriality constraints have an obvious generalization to the multivariate case.

\subsection*{Multivariate polynomial functors}
An $n$-ary, polynomial functor is still defined with a shape $A$ as before, but the content now maps $a ∈ A$ not to a single type, but to a vector of $n$ types.

\begin{leancode}
    structure MvPFunctor (n : Nat) := (A : Type u) (B : A → TypeVec.{u} n)
\end{leancode}

The generalization proceeds relatively straightforwardly.
\begin{leancode}
    def MvPFunctor.Obj (P : MvPFunctor.{u} n) : TypeFun.{u,u} n
        := fun (α : TypeVec n) => Σ a : P.A, P.B a ⟹ α

    def MvPFunctor.map  (P : MvPFunctor.{u} n) 
                        (f : v₀ ⟹ v₁) 
                            : P.Obj v₀ → P.Obj v₁ 
        := fun ⟨a, g⟩ => ⟨a, TypeVec.comp f g⟩
\end{leancode}

Where \lean{TypeVec.comp} is the pointwise composition of two vectors of functions.

\begin{remark}
    Implicit variables are not limited to types, we will use \lean{n} and \lean{m} for natural numbers and \lean{v, v₀, v₁, ...} for type vectors.
\end{remark}

\subsection*{Multivariate M- and W-types}

The multivariate W-type is a fixed point with respect to only the last variable. 
Suppose $P$ is an $n+1$-ary polynomial functor given by shapes $A$ and contents $B$, then its W-type is itself an $n$-ary polynomial functor.

The shape of $W$ is given by data-less, well-founded trees, whose nodes are labelled with $a ∈ A$ and whose children are indexed by $last(B_a)$, the last type in the vector of types $B_a$.
That is, the shapes of $W$ are elements of the univariate W-type given by $A$ and $last(B_a)$.

The content for a given tree, then, is given by valid paths to the nodes in this tree.
\begin{leancode}
  /-- A path from the root of a tree to one of its nodes -/
  inductive WPath : P.last.W → Fin2 n → Type u
  | root (a : P.A) (f : P.last.B a → P.last.W) (i : Fin2 n) (c : P.drop.B a i) :
      WPath ⟨a, f⟩ i
  | child (a : P.A) (f : P.last.B a → P.last.W) (i : Fin2 n) (j : P.last.B a) 
          (c : W_path (f j) i) :
      WPath ⟨a, f⟩ i
\end{leancode}
Note that \lean{P.last} denotes the (univariate) polynomial functor given by $A$ and $last(B_a)$, the last type in vector $B_a$, and 
\lean{P.drop} is the (multivariate) polynomial functor given by $A$ and $drop(B_a)$, the result of removing the last type of the vector $B_a$.

To conclude, an element of the multivariate W-type is a pair of a tree and a function from (paths to) nodes in that tree to the data contained at that node.
\begin{leancode}
  def Wp (P : MvPFunctor (n+1)) : MvPFunctor n := 
    { A := P.last.W, B := P.WPath }
\end{leancode}

The M-type is defined analogously, except, the shapes are non-well-founded trees.
That is, given by a univariate M-type.
\begin{leancode}
  def Mp (P : MvPFunctor (n+1)) : MvPFunctor n := 
    { A := P.last.M, B := P.WPath }
\end{leancode}




\subsection*{Multivariate QPFs}
The definition of multivariate QPFs is not very different from the univariate case. We merely replace (polynomial) functors with multivariate (polynomial) functors.

\begin{leancode}
  class MvQpf {n : ℕ} (F : TypeFun.{u,_} n) [MvFunctor F]  where
    P         : MvPFunctor.{u} n
    abs       : ∀ {v}, P.Obj v → F v
    repr      : ∀ {v}, F v → P.Obj v
    abs_repr  : ∀ {v} (x : F v), abs (repr x) = x
    abs_map   : ∀ {v₀ v₁} (f : v₀ ⟹ v₁) (p : P.Obj v₀), 
                    abs (f <$$> p) = f <$$> abs p
\end{leancode}

Where \lean{f <\$\$> p} is the multivariate analogue of \lean{f <\$> p}, i.e., applying 
\lean{(F.map f)} to \lean{p}.

Again, the initial algebra or final coalgebra of a multivariate QPF
can be constructed as a quotient of the underlying polynomial functor's W- or M-type.


\section{Inductive families}%
\label{sec:ind_families}

It is important to clarify that we're only considering inductive \emph{types}, 
for which the recursive occurrences of the type being declared must not use other values for the type parameters. Lean also has inductive \emph{families} of types for which this restriction does not hold.

Consider, for example, \lean{Fin2}, the type of natural numbers less than $n$.
\begin{badleancode}
    inductive BadFin2 (n : Nat)
    | fz : BadFin2 (n+1)
    | fs : BadFin2 n → BadFin2 (n+1)
\end{badleancode}

This won't compile, because this defines an inductive type, and the constructors mention \lean{BadFin2 (n+1)}, while they are only allowed to mention \lean{BadFin2 n}.

The syntax for an inductive family is very similar; we can promote \lean{n} from parameter to \emph{index} by removing the binder, and giving a type signature.

\begin{leancode}
    inductive Fin2 : Nat → Type
    | fz : Fin2 (n+1)
    | fs : Fin2 n → Fin2 (n+1)
\end{leancode}

Inductive families do not correspond to initial algebras in the same way that inductive types do, and the constructions as QPFs \emph{fundamentally} don't support inductive families, nor the coinductive analogue.
Hence, we shall limit ourselves to just (co)inductive types.
