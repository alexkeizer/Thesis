\chapter{Introduction}%
\label{ch:intro}


Modern logic is primarily built on a framework of induction.
It is no surprise, then, that Lean---an interactive theorem prover / dependently typed, functional programming language---prominently features induction.

In Lean, new datatypes are generally defined using the \inductive{} keyword, which exposes a high-level, definitional syntax.

\begin{leancode}
  inductive List α 
    | nil  : List α
    | cons : α → List α → List α
\end{leancode}

Read as follows: \lean{List} has two \emph{constructors}, i.e., ways to construct a list.
\lean{List.nil} is a constant
representing the empty list; 
\lean{List.cons head tail}, where \lean{head} is of type \lean{α} and \lean{tail} is another list, representing the operation of adding a new element to the front of an existing list. Notice also that \lean{α} is a type parameter, meaning that the list is generic over the type of its elements; \lean{List Nat} is a list of natural numbers, while \lean{List String} is a list of strings.

From this description, the datatype is freely generated, as codified by the characteristic principles that are automatically added by the \inductive{} command.
The \emph{principle of no confusion} states that different combinations of constructors yield distinct elements of the inductive type, and the \emph{recursion principle} embodies structural recursion on that type. For instance, the following is a simplified signature of a recursion principle for lists:
\begin{center}
    \lean{rec : β → (α → β → β) → (List α → β)}
\end{center}
Where the function \lean{rec b f} maps \lean{List.nil} to the constant \lean{b}, and \lean{List.cons head tail} to \lean{f head (rec f tail)}. Since Lean is dependently typed, the actual recursion principle is slightly more involved, but the core idea is the same.
This principle also asserts that \lean{List.nil} and \lean{List.cons} are the only ways to obtain elements of type \lean{List}. This extra assertion is also sometimes presented separately as the \emph{principle of no junk}.


The inductive interpretation means that every element of \lean{List} consists of only
finitely many applications of its constructors. In particular, it is not possible
to construct a list with an infinite chain of \text{cons} applications
%LEANIGNORE
\begin{center}
    \lean{cons 0 (cons 1 (cons 2 (cons 3 (cons 4 (cons 5 ...)))))}
\end{center}
The recursion must end in \lean{nil} at some point; \lean{List} represents only finite lists.
That being said, (countably) infinite lists (also called streams) have a straightforward encoding in Lean, as functions from natural numbers to the parameter type.
\begin{leancode}
    def Stream α := Nat → α
\end{leancode}

The type of potentially infinite lists is simply the sum of \lean{List} and \lean{Stream}.
\begin{leancode}
  inductive CoList α
    | fin : List α      -- a finite list, or
    | inf : Stream α    -- an infinite list
\end{leancode}

However, Lean lacks a comprehensive, definitional solution to defining infinite data structures.

To address this, Avigad \etal{}\ describe a framework for defining \emph{coinductive} types (also called codatatypes) in Lean,
basing their constructions on the category theoretical notion of \emph{quotients of polynomial functors} (QPF).
They also provide a formalization of their work in Lean 3~\cite{avigadDataTypesQuotients2019}.
However, to define a (co)datatype using this framework directly, a user has to hand-compile their specification 
into the provided fundamental operations, making it a not very user-friendly way to define a type.

The main contribution of this thesis is the description, and prototype implementation, of \data{} and \codata{} commands, enabling users to write a (co)datatype specification with the familiar, and more convenient, syntax of \inductive{}, which is then automatically compiled into the appropriate constructions on QPFs.
The implementation of these commands was done in Lean 4; I ported the existing formalization by Avigad \etal\ to this newer version of Lean.


For example, the following two \codata{} specifications are compiled into coinductive types for infinite lists and potentially infinite lists, equivalent to the ad hoc definitions we saw above.

\begin{leancode}
  codata Stream' α
    | cons : α → Stream' α → Stream' α 

  codata CoList' α 
    | nil  : CoList' α
    | cons : α → CoList' α → CoList' α
\end{leancode}

Similarly, we can redefine inductive (i.e., finite) lists in terms of \data{}.

\begin{leancode}
  data List' α 
    | nil  : List' α
    | cons : α → List' α → List' α
\end{leancode}


The QPF approach is \emph{compositional}:
The \data{} and \codata{} commands recognize when the newly defined (co)datatype is itself a QPF, and automatically register it as such, enabling its use in subsequent (co)datatype definitions.



This is not limited to just induction or coinduction; (co)datatypes may be defined with a nested mix of both.
For example, the following defines a potentially infinitely branching (because children of a node are stored in a \lean{CoList}) tree of finite depth (because of the inductive interpretation of \data{}).

\begin{leancode}
  data RoseTree α
    | node : α → CoList' (RoseTree α) → RoseTree α
\end{leancode}

If, instead, we define a coinductive type, we would get trees of infinite depth.
\begin{leancode}
  codata CoRoseTree α β
    | node : α → CoList' (CoRoseTree α) → CoRoseTree α
\end{leancode}

We can also define finitely branching trees of infinite depth, by replacing \lean{CoList'} with \lean{List'}.
\begin{leancode}
  codata CoRoseTree₂ α β
    | node : α → List' (CoRoseTree₂ α) → CoRoseTree₂ α
\end{leancode}


\subsection*{Quotients}
Besides \inductive{} types, Lean also supports defining new types as quotients of other types.
It is common, e.g., to define a multiset as the equivalence class of lists with respect to the relation that equates lists up to permutation.

%LEANIGNORE
\begin{leancode}
  /-- `List.perm as bs` holds iff `as` is a permutation of `bs` -/
  def List.perm : List α → List α → Prop

  def Multiset α := Quot.mk (@List.perm α)
\end{leancode}

We'd like to be able to use \lean{Multiset} to define the type of unordered trees. 

\begin{badleancode}

  inductive UnorderedTree α
    | node : α →  Multiset (UnorderedTree α) → UnorderedTree α
    
\end{badleancode}

Yet, this definition won't compile; it is not allowed to nest a recursive occurrence of \lean{UnorderedTree}
behind a quotient type. Lean's built-in inductive and quotient types do not compose well.


As the name suggests, QPFs support a notion of quotient types. Our datatype package does not yet
provide user-friendly syntax for defining QPF-based quotient types, but suppose that \lean{Multiset'} were manually
defined in terms of QPFs, then we can define unordered trees as follows:
\begin{leancode}
  noncomputable data UnorderedTree' α
    | node : α →  Multiset' (UnorderedTree' α) → UnorderedTree' α
\end{leancode}

That is, the compositionality of QPFs extends to arbitrary mixes of inductive, coinductive and quotient types.




\subsection*{Limitations}
The implementation is not complete yet, and in particular, doesn't yet generate nicely encapsulated no confusion and (co)recursion principles.
The QPF framework certainly allows for these principles to be derived, and it is already possible to define (co)recursive functions on (co)datatypes.
However, only by invoking the fundamental operation, exposing the supposedly internal QPF encoding. 

A major opportunity for future work lies in improving this situation, bringing the experience of writing (co)recursive for \data{} and \codata{} types closer to the equational approach used to define recursive functions for standard \inductive{} types.



\subsection*{Related Work}
The QPF framework by Avigad \etal{}, and the algebraic study of datatypes in general, is based on the observation that datatypes are generally \emph{functorial}~\cite{avigadDataTypesQuotients2019}.

Specifically, the QPF constructions are a variant of the notion of \emph{bounded natural functors} (BNF), 
which dates back to an effort to add a definitional package for (co)datatypes to Isabelle---a different theorem prover~\cite{biendarraDefiningCoDatatypes,traytelCategoryTheoryBased}.
Inspiration has also gone the other direction, with Fürer \etal{} studying quotients of BNFs~\cite{furerQuotientsBoundedNatural2022}.


Lean is developed by de Moura \etal{} at Microsoft Research, based on the Calculus of Inductive Constructions~\cite{demouraLeanTheoremProver2015,coquandMetamathematicalInvestigationsCalculus1989}.
Agda and Coq, two more theorem provers, provide support
for coinductive types and corecursion directly in their trusted kernel~\cite{gimenezApplicationCoinductiveTypes1996,gimenezTutorialRecursiveTypes1998}. Contrast this with the BNF or QPF approaches, which are implemented entirely as a
library.

Basold~and~Geuvers~\cite{basoldMixedInductiveCoinductiveReasoning2018, basoldTypeTheoryBased2016}
studied a dependent type theory with coinductive types separate from any specific theorem prover.
Their approach ensures a computational meaning of the terms.


% \begin{todo}
%     Expand on related work section
% \end{todo}


\subsection*{Organization}
\Cref{ch:background} will explain what QPFs are, and illustrate Lean's syntax in the process.
\Cref{ch:porting} will dig into the differences between Lean 3 and Lean 4, and detail the process of porting the QPF formalizations made by Avigad \etal{}
\Cref{ch:enhancing} will describe enhancements made to the formalizations in the process, which go beyond just porting the existing behaviour.
\Cref{ch:procedure} will establish the procedure to translate the definitional syntax of \data{} and \codata{} into the proper constructions in the theory of QPFs.
\Cref{ch:implementing} will go into technical detail about the (proof of concept)
implementation of these commands, and the Lean 4 meta-programming system.
\Cref{ch:limitations} will discuss the limitations of the current implementation, and opportunities for improvement.
Finally, \cref{ch:conclusion} concludes the thesis.

Accompanying code can be found at \url{https://github.com/alexkeizer/qpf4}.
It provides the \data{} and \codata{} commands discussed so far, alongside a \qpf{} command for the composition
pipeline. A preconfigured environment is also made available at \url{https://gitpod.io#github.com/alexkeizer/qpf4-example/blob/main/Scratch/Examples.lean}, 
allowing readers to try out \data{} and \codata{} in their browser, with minimal configuration.

The code snippets in this thesis are tested with version \lean{leanprover/lean4:nightly-2022-04-28}. The same version was used to develop the code in the linked repository and runs in the preconfigured environment.
The thesis also contains code that (intentionally) does not typecheck. These snippets are typeset with a red line, like so:
\begin{badleancode}

    def Foo := Bar -- whoops, Bar does not exist

\end{badleancode}