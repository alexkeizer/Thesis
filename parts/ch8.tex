
\chapter{Conclusion}%
\label{ch:conclusion}

In this thesis, I presented the procedure behind, and implementation of, a (co)datatype package for Lean 4.
The package enables users to ergonomically define coinductive types to Lean 4 in a compositional way,
with support for mixed induction, coinduction, and quotients.

The package is based on the category theoretical notion of a quotient of polynomial functor.
An existing formalization of QPFs and their constructions was ported from Lean 3 to Lean 4.
The updated formalizations will eventually be made available as part of the mathlib4 project, the port of Leans community-maintained, comprehensive mathematics library.

The prototype implementation makes extensive use of Leans meta-programming system.
Specifically, the \data{} and \codata{} commands accept the same syntax as Leans standard \inductive{} type specifications.
We also discuss the details of elaborating both commands in Lean, and weighed the trade-offs involved with 
synthesizing new declarations.



That being said, defining a type is only part of the equation. More work is needed to simplify the definition of (co)recursive functions. We have briefly outlined some strategies that future work might pursue.


