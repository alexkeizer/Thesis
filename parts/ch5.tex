
\chapter{A procedure for synthesizing functors from a specification}
\label{ch:procedure}

This chapter will establish a procedure that transforms a specification of a (co)datatype
into the proper constructions on QPFs.
It will do so in the abstract, focusing on the details of the procedure, rather than implementation
details of the Lean meta-programming system (which will be covered in the next chapter).


\section{Shape types}
\label{sec:shape_types}

Arguably the simplest, and most fundamental, inductive types are \leanm{Sum α β} and \leanm{Prod α β},
representing "either α or β" and "a pair of α and β", respectively.
They can be defined as

\begin{center}
\begin{minipage}[t]{0.45\linewidth}
    \begin{leancode}
inductive Sum α β
  | inl : α → Sum α β
  | inr : β → Sum α β
    \end{leancode}
\end{minipage}
\begin{minipage}[t]{0.45\linewidth}
    \begin{leancode}    
inductive Prod α β
  | mk : α → β → Prod α β
    \end{leancode}
\end{minipage}
\end{center}

They are also examples of what we will call \emph{shape} types.
\begin{definition}
    A \emph{shape} type is an inductive type \leanm{Foo α_1, ..., α_n}, 
    where each constructor takes only arguments of types in $\{α_1, ..., α_n\}$.
\end{definition}
That is, each constructor's arguments must be typed as one of the parameters to the shape type.
Let's make this a bit clearer by look at examples that are \textbf{not} shape types. 

%LEAN namespace BadExample
\begin{leancode}
  inductive List α
    | nil  : List α 
    | cons : α → List α → List α

  inductive ListWrapper α
    | mk : List α → ListWrapper α

  inductive NatWrapper
    | mk : Nat → NatWrapper
\end{leancode}
%LEAN end BadExample

The only parameter to \leanm{List} is α, but the \leanm{cons} constructor takes a \leanm{List α} as second argument,
so \leanm{List} is not a shape type.
Similarly, \leanm{ListWrapper.mk} (resp. \leanm{NatWrapper.mk}) takes an argument of type \leanm{List α} (resp. \leanm{Nat}),
which are not type parameters, so these types are not shapes either.

Notice that shape types are non-recursive and do not depend on any other types, as a direct consequence 
of the definition. This makes them easy to translate into a polynomial functor. 

\begin{remark}
    One way to do this translation is to realize that all shape functors can be defined as a
    composition of sums and products. 
    This is similar to what the datatype package for Isabelle/HOL in~\cite{traytelCategoryTheoryBased} does.
    We'll use a different, slightly more monolithic approach.
\end{remark}

Recall that (multivariate) polynomial functors are defined as
\begin{leancode}
    structure MvPFunctor (n : Nat) :=
      (A : Type u) (B : A → TypeVec n)
\end{leancode}


Let us return to the example of sum and product types. 
For the ``head'' type (\leanm{A}), we will take a type that has exactly as many constructor as
the shape type, but such that each constructor is a constant (i.e., takes no arguments).
Note that the head type does not take any type parameters. 

\begin{center}
    \begin{minipage}[t]{0.45\linewidth}
        \begin{leancode}
    inductive Sum.HeadT
      | inl : HeadT
      | inr : HeadT
        \end{leancode}
    \end{minipage}
    \begin{minipage}[t]{0.45\linewidth}
        \begin{leancode}    
    inductive Prod.HeadT
      | mk : HeadT
        \end{leancode}
    \end{minipage}
\end{center}

The ``child'' family of types maps each constructor $c$ to a vector of types \leanm{α_c}.
What is most important is the cardinality of each type \leanm{α_c i}, 
because that is what determines the number of arguments of the $i$-th type parameter are needed
to use constructor $c$. 

The concrete structure of these types is not relevant, so we'll always use \leanm{PFin2 m} (see \cref{sec:enhance:pfin2}), the type
of natural numbers less than $m$, to construct a type with cardinality $m$.

\begin{remark}
    We could have used \lean{PFin2} for the head type as well, rather than generating bespoke inductive types.
    However, the current approach has the benefit that it's very clear which element of \lean{HeadT} represents which
    constructor.
\end{remark}

Let us start by counting for each constructor and each parameter type, how many times the constructor takes an argument of that type.
\begin{figure}[h]
    \begin{center}
        \begin{tabular}{l|c|c}
            & α & β \\ \hline
            \lean{Sum.inl}  & 1 & 0 \\
            \lean{Sum.inr}  & 0 & 1 \\
            \lean{Prod.mk}  & 1 & 1 \\    
        \end{tabular}    
    \end{center}

     \caption{Constructor argument bookkeeping}%
    \label{fig:ctor_bookkeepping}
\end{figure}


Using the counts of \cref{fig:ctor_bookkeepping}, we define the child family of types, and subsequently, the polynomial functor.

\begin{center}
  \begin{leancode}
    def Sum.ChildT : Sum.HeadT → TypeVec 2
      | .inl => ![PFin2 1, PFin2 0]
      | .inr => ![PFin2 0, PFin2 1]

    def Prod.ChildT : Prod.HeadT → TypeVec 2
      | .mk  => ![PFin2 1, PFin2 1]
  \end{leancode}
\end{center}

\begin{remark}
    If Lean knowns which type to expect, say \lean{Sum.HeadT}, and we write an identifier with a leading
    dot, like \lean{.inl}, then it will automatically add the type as namespace, concluding that
    we must mean \lean{Sum.HeadT.inl}.
\end{remark}
From here on, the construction is the same for both types;
We'll show it just for \lean{Sum}.

\begin{leancode}
    def Sum.P  : MvPFunctor 2 := MvPFunctor.mk Sum.HeadT  Sum.ChildT
    
    def QpfSum.Uncurried : TypeFun 2   := MvPFunctor.Obj Sum.P
    def QpfSum : Type → Type → Type := TypeFun.curried QpfSum.Uncurried
\end{leancode}

And we're done! However, these qpf-based versions of the types are still not very nice to use.
For example, if we want to construct a pair in \leanm{QpfProd}, we have to go through \leanm{MvPFunctor.mk},
which encodes its arguments in a not user-friendly way. 
Namely, to construct \lean{Sum α β} from an \lean{(a : α)}, i.e., use the \lean{inl} constructor, 
it expects something of type
\begin{center}
    \leanm{(Sum.ChildT .inl) ⟹ ![α, β]}
\end{center}
Which is shorthand for, 
\begin{center}
    (i : PFin 2) → (\leanm{![PFin2 1, PFin 0] i → ![α, β] i})
\end{center}
That is, a function \leanm{PFin2 1 → α} and a function \leanm{PFin2 0 → β}.
\begin{remark}
    Note that vectors are indexed right-to-left, so \lean{![α, β] 0} is β
    and \lean{![α, β] 1} is α
\end{remark}

Recall that we defined vectors of size $n$ as functions \lean{PFin2 n → α}, so 


\begin{leancode}
    def QpfSum.inl {α β} (a : α) : QpfSum α β :=
      MvPFunctor.mk .inl (fun (i : PFin 2) => match i with
        | 1 => ![a]
        | 0 => ![]
      )
    
    def QpfSum.inr {α β} (b : β) : QpfSum α β :=
      MvPFunctor.mk .inr (fun (i : PFin 2) => match i with
        | 1 => ![]
        | 0 => ![b]
      )
\end{leancode}


We could expose these constructors, but the inner details will still be exposed
when a user tries to destruct an element of \lean{QpfSum α β}.
Clearly, this is not an ideal definition.

\subsection{MvQpf.ofPolynomial}
The solution is to not reinvent the wheel quite as much.
The \leanm{inductive} version of \lean{Sum} and \lean{Prod} works perfectly fine,
since shape types are non-recursive and non-composite by definition.

In fact, the goal of this part of the procedure is not to redefine shape types, it is to show that
they are QPFs by deriving an instance of \lean{MvQpf}.

It stands to reason that if $P$ is a polynomial functor, and $F$ is isomorphic to $P$, then $F$ is at
the very least a QPF, this is formalized by \lean{MvQpf.ofPolynomial}.

%LEANIGNORE
\begin{leancode}
    def ofPolynomial {F : TypeFun n} 
                     (P : MvPFunctor n) 
                     (box    : ∀{α}, F α → P.Obj α) 
                     (unbox  : ∀{α}, P.Obj α → F α) 
                     (box_unbox_id : ∀{α} (x : P.Obj α), box (unbox x) = x)
                     (unbox_box_id : ∀{α} (x : F α), unbox (box x) = x)
                  : MvQpf F
\end{leancode}

In the case of our \lean{Sum} example, we're instantiating \lean{ofPolynomial} with 
\lean{F := TypeFun.ofCurried Sum} and \lean{P := Sum.P}.

To generate \lean{box} and \lean{unbox} we expand our bookkeeping a bit. For each constructor argument
we generate a fresh identifier, and while doing so we keep two lists: a list of all identifiers, in the
order they were introduced, and a separate list for each parameter type, with just the identifiers for that type.

To illustrate, consider the following, slightly artificial, type
\begin{leancode}
  inductive SumOfPairs α β γ where
    | pairA     : α → α → SumOfPairs α β γ 
    | pairCandB : γ → β → SumOfPairs α β γ   -- note the order of β and γ
\end{leancode}

Following the procedure introduced at the start of this section, we obtain a corresponding polynomial functor.
\begin{leancode}
  inductive SumOfPairs.HeadT where
    | pairA     : HeadT
    | pairCandB : HeadT

  def SumOfPairs.ChildT : SumOfPairs.HeadT → TypeVec 3
    | .pairA      => ![PFin2 2, PFin2 0, PFin2 0]
    | .pairCandB  => ![PFin2 0, PFin2 1, PFin2 1]

  def SumOfPairs.P : MvPFunctor 3 :=
    ⟨SumOfPairs.HeadT, SumOfPairs.ChildT⟩
\end{leancode}

Let us use \lean{a₀} and \lean{a₁} for the arguments to the first constructor, and \lean{c} and \lean{b}
for the second constructor, then we separate the arguments by their type.

\begin{figure}[h]
\begin{center}
    \begin{tabular}{c|c|c}
                    & \lean{pairA}  & \lean{pairCandB}  \\ \midrule
        \emph{all}  & \lean{[a₀, a₁]} & \lean{[c, b]}   \\ \midrule
        \lean{α}    & \lean{[a₀, a₁]} & \lean{[]}       \\
        \lean{β}    & \lean{[]} & \lean{[b]}       \\
        \lean{γ}    & \lean{[]} & \lean{[c]}       \\
    \end{tabular}
\end{center}

\caption{Identifier bookkeeping for \texttt{\small SumOfPairs} constructors}%
\label{fig:fresh_ctor_vars}    
\end{figure}

The definition of \lean{box} follows fairly directly from \cref{fig:fresh_ctor_vars}, namely
\begin{leancode}
  def SumOfPairs.box : SumOfPairs α β γ → SumOfPairs.P.Obj ![α, β, γ]
    -- we pattern match using all variables for this constructor
    | .pairA a₀ a₁ => MvPFunctor.mk .pairA fun i => match i with
      | 2 => ![a₀, a₁]  -- α
      | 1 => ![]        -- β 
      | 0 => ![]        -- γ
    | .pairCandB c b => MvPFunctor.mk .pairCandB fun i => match i with
      | 2 => ![]        -- α 
      | 1 => ![b]       -- β 
      | 0 => ![c]       -- γ
\end{leancode}

Logically, \lean{unbox} does the reverse, 
\begin{leancode}
    def SumOfPairs.unbox : SumOfPairs.P.Obj ![α, β, γ] → SumOfPairs α β γ
      := fun ⟨head, child⟩ => match head with
          | .pairA      => SumOfPairs.pairA (child 2 0) (child 2 1)
          | .pairCandB  => SumOfPairs.pairCandB (child 0 0) (child 1 0)
  \end{leancode}

The proof that these two functions are indeed inverses is a straightforward mix of case distinction and reflexivity.






\section{Recursive and corecursive types}%
\label{sec:rec_types}
If a type is recursive, but otherwise does not mention other types it is not strictly a shape type,
but we can transform it into one.
Namely, by adding an extra type parameter \lean{ρ} and substituting \lean{ρ} for all (co)recursive occurences of the type
to be defined.

For example, the shape of \lean{List α} (see \cref{sec:shape_types}) is
\begin{leancode}
  inductive List.Shape α ρ
    | nil  : Shape α ρ
    | cons : α → ρ → Shape α ρ
\end{leancode}
This is a valid shape type, so we follow the procedure above to derive an instance of \lean{MvQpf}.
To get rid of the extra variable \lean{ρ}, we simply take the fixpoint.
\begin{leancode}
  def QpfList.Uncurried : TypeFun 2 
    := MvQpf.Fix (TypeFun.ofCurried MyList.Shape)
\end{leancode}

Conversely, suppose we wish to define \lean{CoList}, the \emph{coinductive} type of potentially infinite lists.
We introduced this type in the introduction with the following specification.
\begin{leancode}
  inductive CoList α
    | nil  : CoList α
    | cons : α → CoList α → CoList α
\end{leancode}
There is no difference in the procedure to obtain the corresponding shape, we just replace all occurences
of \lean{CoList α} as constructor argument type with a new type parameter \lean{ρ}, obtaining the exact same shape
as for \lean{List}.
\begin{leancode}
  inductive CoList.Shape α ρ
    | nil  : Shape α ρ
    | cons : α → ρ → Shape α ρ
\end{leancode}
But rather than taking the fixpoint, we take the \emph{cofixpoint} for a coinductive interpretation.
\begin{leancode}
  def CoList.Uncurried : TypeFun 2 
    := MvQpf.CoFix (TypeFun.ofCurried CoList.Shape)
\end{leancode}





\section{Composition pipeline}%
\label{sec:comp_pipeline}
The running example for this section will be the rose tree; leaves are labelled with \lean{α}, while
internal nodes are labelled with \lean{β} and can have an finite, non-zero number of children.
\begin{leancode}
  data QpfTree α β
    | leaf : α → QpfTree α β
    | node : β → QpfTree α β → QpfList (QpfTree α β) → QpfTree α β
\end{leancode}

The type is recursive, so we introduce the fresh parameter as before.
\begin{leancode}
  data QpfTree.Nonrecursive α β ρ
    | leaf : α → QpfTree α β ρ
    | node : β → ρ → QpfList ρ → QpfTree α β ρ
\end{leancode}

However, the result is not quite a shape type yet, we also have to get rid of \lean{QpfList ρ} as an
argument type. To do so, we simply introduce more parameters, while remembering which type these
new parameters are supposed to stand for.

\begin{leancode}
  inductive QpfTree.Shape α β ρ σ₁
    | leaf : α → QpfTree α β ρ σ₁
    | node : β → ρ → σ₁ → QpfTree α β ρ σ₁
\end{leancode}

Then, proceed as in \cref{sec:shape_types} to show that \lean{QpfTree.Shape} is a QPF.

\begin{remark}
  When doing this substitution, we can reuse the same parameter for multiple occurrences of the same type.
  Suppose we had a constructor that takes two list, like
  %LEANIGNORE
  \begin{leancode}
    | node₂ : β → ρ → QpfList ρ → QpfList ρ → QpfTree α β ρ
  \end{leancode}
  Then we can reuse the same fresh parameter \lean{σ₁} for both occurrences.
  %LEANIGNORE
  \begin{leancode}
    | node₂ : β → ρ → σ₁ → σ₁ → QpfTree α β ρ σ₁
  \end{leancode}
  
  On the other hand, if a non-parameter type also occurs as a subexpression of another type, then
  we will not substitute it with the same parameter.
  The example gets a bit contrived, but suppose nodes take both a list of children, and a nested list 
  of lists of children.
  %LEANIGNORE
  \begin{leancode}
    | node₃ : β → ρ → QpfList ρ → QpfList (QpfList ρ) → QpfTree α β ρ
  \end{leancode}
  This is translated to
  %LEANIGNORE
  \begin{leancode}
    | node₃ : β → ρ → σ₁ → σ₂ → QpfTree α β ρ σ₁ σ₂
  \end{leancode}
  Where \lean{σ₁} stands for \lean{QpfList ρ}, as before, and \lean{σ₂} stands for \lean{QpfList (QpfList ρ)}, \emph{not} \lean{QpfList σ₁}.    
  
  Do note that this does \emph{not} apply when we are adding a new parameter (such as \lean{ρ} in our example) for recursive occurences in the very first step (to obtain the \lean{Nonrecursive} specification), such variables \emph{do} get substituted in all subexpressions.
  \end{remark}

Returning to \lean{QpfTree.Shape}, parameter \lean{σ₁} is supposed to stand for \lean{QpfList ρ}, so we're aiming to define a QPF \lean{Base} which satisfies:
%LEANIGNORE
\begin{leancode}
  Base α β ρ = QpfTree.Shape α β ρ (QpfList ρ)
\end{leancode}


The \emph{composition pipeline} translates such an equation to a definition of \lean{Base} in terms of the appropriate construction on QPF, such that: (a) \lean{Base} is known to be a QPF, and (b) the desired equality indeed holds.
The composition pipeline is not just an internal detail of the procedure, we also expose it as the \qpf{} macro, whose syntax closely matches that of \keyword{def}.
\begin{leancode}
  qpf Base α β ρ := QpfTree.Shape α β ρ (QpfList ρ)
\end{leancode}



As the name ``composition pipeline'' alludes, we are interested in defining compositions of QPFs. Formally, \lean{MvQpf.Comp} has the signature:
\begin{center}
    \lean{TypeFun n → Vec (TypeFun m) n → TypeFun m}
\end{center}
That is, an $n$-ary type function \lean{F} is composed with an $n$-sized vector of $m$-ary type functions, resulting in an $m$-ary type function. The composition is essentially defined as
%LEANIGNORE
\begin{leancode}
    (Comp F ![G₁, ..., Gₙ]) ![α₁, ..., αₘ] 
                = F ![G₁ ![α₁, ..., αₘ], ..., Gₘ ![α₁, ..., αₘ]]
\end{leancode}
All arguments \lean{αᵢ} are broadcast to all functors \lean{Gⱼ}, meaning we don't have to worry about argument duplication or reordering.

Continuing with the motivating example, we are going to define \lean{Base} as \lean{Comp QpfTree.Shape ![G₁, G₂, G₃, G₄]}, for some functors \lean{G₁}, \lean{G₂}, \lean{G₃} and \lean{G₄}, satisfying the following equalities:
\begin{leancode}
  qpf G₁ α β ρ := α                qpf G₃ α β ρ := ρ
  qpf G₂ α β ρ := β                qpf G₄ α β ρ := (QpfList ρ)
\end{leancode}

Which we will recursively determine.
Before we continue, it should be clarified that the composition pipeline does not support any kind
of equation. For example, it is impossible to define a QPF that satisfies
\begin{badleancode}
  qpf H α β := α → β 
\end{badleancode}
Because the arrow type constructor \lean{(⋅ → ⋅)} is \emph{not} functorial. 
At least, it is not functorial in both arguments. If, instead, we fix any value \lean{α} for the
first argument, we obtain \lean{(α → ⋅)}, which \emph{is} a QPF.
More generally, we say that in \lean{α → β}, α is a \emph{dead} variable, since it occurs as a non-functorial argument, while β is a \emph{live} variable.

By default, the composition pipeline assumes arguments are live, but we can explicitly mark some as dead
by giving a type ascription. 
\begin{leancode}
  qpf H (α : Type u) β = α → β 
\end{leancode}
The result is a family of QPFs, which means \lean{H} is not a QPF, but \lean{H a}, for arbitrary values \lean{a : α} is.




Continuing on, the composition pipeline does support these three kinds of functors:
\begin{itemize}
    \item \emph{Projections: } The right-hand-side of the equation is just a parameter, as in \lean{G₁ α β ρ = α}
    \item \emph{Constants: } The rhs does not mention the live parameters at all, e.g., \lean{G α β = Nat} or 
                \leanm{G (n : Nat) β = PFin2 n}.
    \item \emph{Compositions: } The rhs is an application of a QPF
\end{itemize}

This notably excludes dependent arrows \lean{(a : α) → \_}, anonymous functions \leanm{fun γ => _},
and applications of non-QPFs (such as non-dependent arrows (⋅ → ⋅) in its general form).



Projections are trivial to identify (see \lean{G₁}, \lean{G₂} and \lean{G₃} from the example before), 
and are represented with \lean{MvQpf.Prj n i}, where $n$ is the arity and $i$ the index
of the parameter to project to (counted right-to-left).
\begin{leancode}
  qpf G₁ α β ρ := α  --> def G₁ := @MvQpf.Prj 3 2
  qpf G₂ α β ρ := β  --> def G₂ := @MvQpf.Prj 3 1
  qpf G₃ α β ρ := ρ  --> def G₃ := @MvQpf.Prj 3 0
\end{leancode}


For constants we only have to verify that the target expression does not depend on live parameters,
then we translate them into \lean{@MvQpf.Const n τ}, with $n$ again the arity and \lean{τ} the constant type. 
\begin{leancode}
  qpf C₁ α β γ       := Nat     --> def C₁           := @MvQpf.Prj 3 Nat
  qpf C₂ (n : Nat) α := PFin2 n --> def C₂ (n : Nat) := @MvQpf.Prj 1 (PFin2 n)
\end{leancode}
In the second example, \lean{PFin2 n} is not strictly a constant, it depends on \lean{n}, but it still
counts, since \lean{n} is marked as a dead parameter.

The last kind of target expression supported is an application, specifically, applications that 
can be broken down into $\lean{G e₁ ... eₖ}$, for a arbitrary arity $k$, where \lean{G} is a $k$-ary QPF.
If there are multiple choices, we pick the option that minimizes $k$. This is the kind of expression we
started this section with, and already discussed how to handle---as \lean{MvQpf.Comp} with recursively determined functors representing arguments \lean{eᵢ}.

As an optimization, we don't have to generate \keyword{def} declarations for all intermediate functors, we can just
inline them.
\begin{leancode}
  qpf G₄ α β ρ := QpfList ρ 
    --> def G₄ α β ρ := MvQpf.Comp QpfList ![@MvQpf.Prj 3 0]

  qpf Base α β ρ := QpfTree.Shape α β ρ (QpfList ρ)
    --> def Base α β ρ := MvQpf.Comp QpfTree.Shape ![
    -->     @MvQpf.Prj 3 2,                       -- G₁ 
    -->     @MvQpf.Prj 3 1,                       -- G₂ 
    -->     @MvQpf.Prj 3 0,                       -- G₃
    -->     MvQpf.Comp QpfList ![@MvQpf.Prj 3 0], -- G₄
    --> ]
\end{leancode}



Finally, the desired type is obtained by taking the fixpoint. Here, too, we can inline the definition
of \lean{Base}.
\begin{leancode}
  def QpfTree α β : TypeFun 2 :=
    MvQpf.Fix (MvQpf.Comp QpfTree.Shape ![
        @MvQpf.Prj 3 2,                       -- G₁ 
        @MvQpf.Prj 3 1,                       -- G₂ 
        @MvQpf.Prj 3 0,                       -- G₃
        MvQpf.Comp QpfList ![@MvQpf.Prj 3 0], -- G₄
    ])
\end{leancode}




\section{Auxiliary Constructions}%
\label{sec:aux_constructions}

That are all the steps for constructing a (co)datatype, but to make that type more useful, there are
a few extra definitions we want to add. To begin, we want to generate a function for each constructor 
in the specification, with the same name and type. Generating these is relatively straightforward.
For example, let us recall a specification for lists.
\begin{leancode}
  data List α
    | nil  : List α 
    | cons : α → List α → List α
\end{leancode}
Following the steps described so far defines \lean{List} in terms of a generated shape type \lean{List.Shape}
and a fixpoint. The constructors we want to generate are simple wrappers around these.
\begin{leancode}
  def List.nil : List α 
    := MvQpf.Fix.mk List.Shape.nil

  def List.cons : α → List α → List α 
    := fun a as => MvQpf.Fix.mk (List.Shape.cons a as)
\end{leancode}
Effectively, we are just composing \lean{Fix.mk} with the corresponding constructor of \lean{List.Shape}.
The same methods applies to codatatypes, using \lean{CoFix.mk} instead.

At this point we would also want to generate specialized (co)recursion principles and \lean{noConfusion}
theorems, but as mentioned in the introduction, we don't support doing so yet.


% TODO: check if this pagebreak still makes sense
% \pagebreak[3]
\section{Final Overview}%
\label{sec:procedure:overview}

To sum up, we can give a high-level overview of the procedure in the following steps.
\begin{enumerate}
  \item\label{enum:proc:make_nonrec}
    Make the specification non-recursive by introducing a new parameter (\cref{sec:rec_types})
  \item\label{enum:proc:mkShape} 
    Introduce more parameters, to obtain a shape type \lean{Shape} (\cref{sec:comp_pipeline})
  \item\label{enum:proc:mkQpf} 
    Show that \lean{Shape} is a QPF, by (\cref{sec:shape_types})
        \begin{enumerate}
          \item Deriving an equivalent polynomial functor, and
          \item Showing that this polynomial functor is isomorphic to \lean{Shape}
        \end{enumerate}
  \item\label{enum:proc:mkBase} 
    Use the composition pipeline to solve for the intended values for parameters introduced in step 2, producing the \lean{Base} functor (\cref{sec:comp_pipeline})
  \item\label{enum:proc:fixpoint} 
    Take the fixpoint (or cofixpoint) of \lean{Base}, getting rid of the parameter introduced in step 1 
    and producing the desired functor (\cref{sec:rec_types})
  \item\label{enum:proc:aux} 
    Generate auxiliary constructions (\cref{sec:aux_constructions})
\end{enumerate}

Steps~\ref{enum:proc:make_nonrec}~and~\ref{enum:proc:fixpoint} can be omitted if the specification is not (co)recursive, and steps~\ref{enum:proc:mkShape}~and~\ref{enum:proc:mkBase} may be omitted
if the input is already a shape type, but it is also fine to forego such analysis and perform all steps anyway.


