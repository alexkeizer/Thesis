
\chapter{A procedure for synthesizing functors from a specification}
\label{ch:procedure}

This chapter will establish a procedure that transforms a specification of a (co)datatype
into the proper constructions on QPFs.
It will do so in the abstract, focusing on the details of the procedure, rather than implementation
details of the Lean meta-programming system (which will be covered in the next chapter).


\section{Shape types}
\label{sec:shape_types}

Arguably the simplest, and most fundamental, inductive types are \leanm{Sum α β} and \leanm{Prod α β},
representing "either α or β" and "a pair of α and β", respectively.
They can be defined as

\begin{center}
\begin{minipage}[t]{0.45\linewidth}
    \begin{leancode}
inductive Sum α β
  | inl : α → Sum α β
  | inr : β → Sum α β
    \end{leancode}
\end{minipage}
\begin{minipage}[t]{0.45\linewidth}
    \begin{leancode}    
inductive Prod α β
  | mk : α → β → Prod α β
    \end{leancode}
\end{minipage}
\end{center}

They are also examples of what we will call \emph{shape} types.
\begin{definition}
    A \emph{shape} type is an inductive type \leanm{Foo α_1, ..., α_n}, 
    where each constructor takes only arguments of types in $\{α_1, ..., α_n\}$.
\end{definition}
That is, each constructor's arguments must be typed as one of the parameters to the shape type.
Let's make this a bit clearer by look at examples that are \textbf{not} shape types. 

%LEAN namespace BadExample
\begin{leancode}
    inductive List α
      | nil  : List α 
      | cons : α → List α → List α

    inductive ListWrapper α
      | mk : List α → ListWrapper α

    inductive NatWrapper
      | mk : Nat → NatWrapper
\end{leancode}
%LEAN end BadExample

The only parameter to \leanm{List} is α, but the \leanm{cons} constructor takes a \leanm{List α} as second argument,
so \leanm{List} is not a shape type.
Similarly, \leanm{ListWrapper.mk} (resp. \leanm{NatWrapper.mk}) takes an argument of type \leanm{List α} (resp. \leanm{Nat}),
which are not type parameters, so these types are not shapes either.

Notice that shape types are non-recursive and do not depend on any other types, as a direct consequence 
of the definition. This makes them easy to translate into a polynomial functor. 

\begin{remark}
    One way to do this translation is to realize that all shape functors can be defined as a
    composition of sums and products. 
    This is similar to what the datatype package for Isabelle/HOL in~\cite{traytelCategoryTheoryBased} does.
    We'll use a different, slightly more monolithic approach.
\end{remark}

Recall that (multivariate) polynomial functors are defined as
\begin{leancode}
    structure MvPFunctor (n : Nat) :=
      (A : Type u) (B : A → TypeVec n)
\end{leancode}


Let us return to the example of sum and product types. 
For the ``head'' type (\leanm{A}), we will take a type that has exactly as many constructor as
the shape type, but such that each constructor is a constant (i.e., takes no arguments).
Note that the head type does not take any type parameters. 

\begin{center}
    \begin{minipage}[t]{0.45\linewidth}
        \begin{leancode}
    inductive Sum.HeadT
      | inl : HeadT
      | inr : HeadT
        \end{leancode}
    \end{minipage}
    \begin{minipage}[t]{0.45\linewidth}
        \begin{leancode}    
    inductive Prod.HeadT
      | mk : HeadT
        \end{leancode}
    \end{minipage}
\end{center}

The ``child'' family of types maps each constructor $c$ to a vector of types \leanm{α_c}.
What is most important is the cardinality of each type \leanm{α_c i}, 
because that is what determines the number of arguments of the $i$-th type parameter are needed
to use constructor $c$. 

The concrete structure of these types is not relevant, so we'll always use \leanm{PFin2 m} (see \cref{sec:enhance:pfin2}), the type
of natural numbers less than $m$, to construct a type with cardinality $m$.

\begin{remark}
    We could have used \lean{PFin2} for the head type as well, rather than generating bespoke inductive types.
    However, the current approach has the benefit that it's very clear which element of \lean{HeadT} represents which
    constructor.
\end{remark}

Let us start by counting for each constructor and each parameter type, how many times the constructor takes an argument of that type.
\begin{figure}[h]
    \begin{center}
        \begin{tabular}{l|c|c}
            & α & β \\ \hline
            \lean{Sum.inl}  & 1 & 0 \\
            \lean{Sum.inr}  & 0 & 1 \\
            \lean{Prod.mk}  & 1 & 1 \\    
        \end{tabular}    
    \end{center}

     \caption{Constructor argument bookkeeping}%
    \label{fig:ctor_bookkeepping}
\end{figure}


Using the counts of \cref{fig:ctor_bookkeepping}, we define the child family of types, and subsequently, the polynomial functor.

\begin{center}
  \begin{leancode}
    def Sum.ChildT : Sum.HeadT → TypeVec 2
      | .inl => ![PFin2 1, PFin2 0]
      | .inr => ![PFin2 0, PFin2 1]

    def Prod.ChildT : Prod.HeadT → TypeVec 2
      | .mk  => ![PFin2 1, PFin2 1]
  \end{leancode}
\end{center}

\begin{remark}
    If Lean knowns which type to expect, say \lean{Sum.HeadT}, and we write an identifier with a leading
    dot, like \lean{.inl}, then it will automatically add the type as namespace, concluding that
    we must mean \lean{Sum.HeadT.inl}.
\end{remark}
From here on, the construction is the same for both types;
We'll show it just for \lean{Sum}.

\begin{leancode}
    def Sum.P  : MvPFunctor 2 := MvPFunctor.mk Sum.HeadT  Sum.ChildT
    
    def QpfSum.Internal : TypeFun 2   := MvPFunctor.Obj Sum.P
    def QpfSum : Type → Type → Type := TypeFun.curried QpfSum.Internal
\end{leancode}

And we're done! However, these qpf-based versions of the types are still not very nice to use.
For example, if we want to construct a pair in \leanm{QpfProd}, we have to go through \leanm{MvPFunctor.mk},
which encodes its arguments in a not user-friendly way. 
Namely, to construct \lean{Sum α β} from an \lean{(a : α)}, i.e., use the \lean{inl} constructor, 
it expects something of type
\begin{center}
    \leanm{(Sum.ChildT .inl) ⟹ ![α, β]}
\end{center}
Which is shorthand for, 
\begin{center}
    (i : PFin 2) → (\leanm{![PFin2 1, PFin 0] i → ![α, β] i})
\end{center}
That is, a function \leanm{PFin2 1 → α} and a function \leanm{PFin2 0 → β}.
\begin{remark}
    Note that vectors are indexed right-to-left, so \lean{![α, β] 0} is β
    and \lean{![α, β] 1} is α
\end{remark}

Recall that we defined vectors of size $n$ as functions \lean{PFin2 n → α}, so 


\begin{leancode}
    def QpfSum.inl {α β} (a : α) : QpfSum α β :=
      MvPFunctor.mk .inl (fun (i : PFin 2) => match i with
        | 1 => ![a]
        | 0 => ![]
      )
    
    def QpfSum.inr {α β} (b : β) : QpfSum α β :=
      MvPFunctor.mk .inr (fun (i : PFin 2) => match i with
        | 1 => ![]
        | 0 => ![b]
      )
\end{leancode}


We could expose these constructors, but the inner details will still be exposed
when a user tries to destruct an element of \lean{QpfSum α β}.
Clearly, this is not an ideal definition.

\subsection{MvQpf.ofPolynomial}
The solution is to not reinvent the wheel quite as much.
The \leanm{inductive} version of \lean{Sum} and \lean{Prod} works perfectly fine,
since shape types are non-recursive and non-composite by definition.

In fact, the goal of this part of the procedure is not to redefine shape types, it is to show that
they are QPFs by deriving an instance of \lean{MvQpf}.

It stands to reason that if $P$ is a polynomial functor, and $F$ is isomorphic to $P$, then $F$ is at
the very least a QPF, this is formalized by \lean{MvQpf.ofPolynomial}.

%LEANIGNORE
\begin{leancode}
    def ofPolynomial {F : TypeFun n} 
                     (P : MvPFunctor n) 
                     (box    : ∀{α}, F α → P.Obj α) 
                     (unbox  : ∀{α}, P.Obj α → F α) 
                     (box_unbox_id : ∀{α} (x : P.Obj α), box (unbox x) = x)
                     (unbox_box_id : ∀{α} (x : F α), unbox (box x) = x)
                  : MvQpf F
\end{leancode}

In the case of our \lean{Sum} example, we're instantiating \lean{ofPolynomial} with 
\lean{F := TypeFun.ofCurried Sum} and \lean{P := Sum.P}.

To generate \lean{box} and \lean{unbox} we expand our bookkeeping a bit. For each constructor argument
we generate a fresh identifier, and while doing so we keep two lists: a list of all identifiers, in the
order they were introduced, and a separate list for each parameter type, with just the identifiers for that type.

To illustrate, consider the following, slightly artificial, type
\begin{leancode}
  inductive SumOfPairs α β γ where
    | pairA     : α → α → SumOfPairs α β γ 
    | pairCandB : γ → β → SumOfPairs α β γ   -- note the order of β and γ
\end{leancode}

Following the procedure introduced at the start of this section, we obtain a corresponding polynomial functor.
\begin{leancode}
  inductive SumOfPairs.HeadT where
    | pairA     : HeadT
    | pairCandB : HeadT

  def SumOfPairs.ChildT : SumOfPairs.HeadT → TypeVec 3
    | .pairA      => ![PFin2 2, PFin2 0, PFin2 0]
    | .pairCandB  => ![PFin2 0, PFin2 1, PFin2 1]

  def SumOfPairs.P : MvPFunctor 3 :=
    ⟨SumOfPairs.HeadT, SumOfPairs.ChildT⟩
\end{leancode}

Let us use \lean{a₀} and \lean{a₁} for the arguments to the first constructor, and \lean{c} and \lean{b}
for the second constructor, then we separate the arguments by their type.

\begin{figure}[h]
\begin{center}
    \begin{tabular}{c|c|c}
                    & \lean{pairA}  & \lean{pairCandB}  \\ \midrule
        \emph{all}  & \lean{[a₀, a₁]} & \lean{[c, b]}   \\ \midrule
        \lean{α}    & \lean{[a₀, a₁]} & \lean{[]}       \\
        \lean{β}    & \lean{[]} & \lean{[b]}       \\
        \lean{γ}    & \lean{[]} & \lean{[c]}       \\
    \end{tabular}
\end{center}

\caption{Identifier bookkeeping for \texttt{\small SumOfPairs} constructors}%
\label{fig:fresh_ctor_vars}    
\end{figure}

The definition of \lean{box} follows fairly directly from \cref{fig:fresh_ctor_vars}, namely
\begin{leancode}
  def SumOfPairs.box : SumOfPairs α β γ → SumOfPairs.P.Obj ![α, β, γ]
    -- we pattern match using all variables for this constructor
    | .pairA a₀ a₁ => MvPFunctor.mk .pairA fun i => match i with
      | 2 => ![a₀, a₁]  -- α
      | 1 => ![]        -- β 
      | 0 => ![]        -- γ
    | .pairCandB c b => MvPFunctor.mk .pairCandB fun i => match i with
      | 2 => ![]        -- α 
      | 1 => ![b]       -- β 
      | 0 => ![c]       -- γ
\end{leancode}

Logically, \lean{unbox} does the reverse, 
\begin{leancode}
    def SumOfPairs.unbox : SumOfPairs.P.Obj ![α, β, γ] → SumOfPairs α β γ
      := fun ⟨head, child⟩ => match head with
          | .pairA      => SumOfPairs.pairA (child 2 0) (child 2 1)
          | .pairCandB  => SumOfPairs.pairCandB (child 0 0) (child 1 0)
  \end{leancode}

The proof that these two functions are indeed inverses is a straightforward mix of case distinction and reflexivity.






\section{Recursive and corecursive types}
If a type is recursive, but otherwise does not mention other types it is not strictly a shape type,
but we can transform it into one.
Namely, by adding an extra type parameter \lean{ρ} and substituting \lean{ρ} for all (co)recursive occurences of the type
to be defined.

For example, the shape of \lean{List α} (see \cref{sec:shape_types}) is
\begin{leancode}
  inductive List.Shape α ρ
    | nil  : Shape α ρ
    | cons : α → ρ → Shape α ρ
\end{leancode}
This is a valid shape type, so we follow the procedure above to derive an instance of \lean{MvQpf}.
To get rid of the extra variable \lean{ρ}, we simply take the fixpoint.
\begin{leancode}
  def QpfList.Internal : TypeFun 2 
    := MvQpf.Fix (TypeFun.ofCurried MyList.Shape)
\end{leancode}

Conversely, suppose we wish to define \lean{CoList}, the \emph{coinductive} type of potentially infinite lists.
We introduced this type in the introduction with the following specification.
\begin{leancode}
  inductive CoList α
    | nil  : CoList α
    | cons : α → CoList α → CoList α
\end{leancode}
There is no difference in the procedure to obtain the corresponding shape, we just replace all occurences
of \lean{CoList α} as constructor argument type with a new type parameter \lean{ρ}, obtaining the exact same shape
as for \lean{List}.
\begin{leancode}
  inductive CoList.Shape α ρ
    | nil  : Shape α ρ
    | cons : α → ρ → Shape α ρ
\end{leancode}
But rather than taking the fixpoint, we take the \emph{cofixpoint} for a coinductive interpretation.
\begin{leancode}
  def CoList.Internal : TypeFun 2 
    := MvQpf.CoFix (TypeFun.ofCurried CoList.Shape)
\end{leancode}





\section{Composition pipeline}%
\label{sec:comp_pipeline}
Finally, we are ready to discuss (co)datatypes that are composed of other (co)datatypes.

The running example for this section will be the rose tree; leaves are labelled with \lean{α}, while
internal nodes are labelled with \lean{β} and can have an finite, non-zero number of children.
\begin{leancode}
    data QpfTree α β
      | leaf : α → QpfTree α β
      | node : β → QpfTree α β → QpfList (QpfTree α β) → QpfTree α β
\end{leancode}

The type is recursive, so we introduce the fresh parameter as before.
\begin{leancode}
    data QpfTree.Nonrecursive α β ρ
      | leaf : α → QpfTree α β ρ
      | node : β → ρ → QpfList ρ → QpfTree α β ρ
\end{leancode}

Then, we go through each argument type which is not just a parameter, and substitute
it with a fresh parameter.

\begin{leancode}
    data QpfTree.Shape α β ρ σ₁
      | leaf : α → QpfTree α β ρ σ₁
      | node : β → ρ → σ₁ → QpfTree α β ρ σ₁
\end{leancode}
Remembering that \lean{σ₁} stands for \lean{QpfList ρ}, so we should define the following 
composition, in such a way that \lean{F} is a QPF.
%LEANIGNORE
\begin{leancode}
    def F α β ρ := QpfTree.Shape α β ρ (QpfList ρ)
\end{leancode}


\subsection{Parameter reuse}
Of course, multiple occurences of the same non-parameter type don't nead a fresh variable each.
Suppose we had a constructor that takes two list, like
%LEANIGNORE
\begin{leancode}
    | node₂ : β → ρ → QpfList ρ → QpfList ρ → QpfTree α β ρ
\end{leancode}
Then we can reuse the same fresh parameter \lean{σ₁} for both occurences.
%LEANIGNORE
\begin{leancode}
    | node₂ : β → ρ → σ₁ → σ₁ → QpfTree α β ρ σ₁
\end{leancode}

On the other hand, if a non-parameter type also occurs as a subexpression of another type, then
we will not substitute it with the same parameter.
The example gets a bit contrived, but suppose nodes take both a list of children, and a nested list 
of lists of children.
%LEANIGNORE
\begin{leancode}
    | node₃ : β → ρ → QpfList ρ → QpfList (QpfList ρ) → QpfTree α β ρ
\end{leancode}
Then we can reuse the same fresh parameter \lean{σ₁} for both occurences.
%LEANIGNORE
\begin{leancode}
    | node₃ : β → ρ → σ₁ → σ₂ → QpfTree α β ρ σ₁ σ₂
\end{leancode}
Where \lean{σ₁} stands for \lean{QpfList ρ}, as before, and \lean{σ₂} stands for 
\lean{QpfList (QpfList ρ)}.



\subsection{Solving for compositions}%
\label{subsec:comp_pipeline:solving}

Returning to our motivating example, recall that we're aiming to define a QPF \lean{F} which satisfies:

%LEANIGNORE
\begin{leancode}
    F α β ρ = QpfTree.Shape α β ρ (QpfList ρ)
\end{leancode}

The \emph{composition pipeline} translates such an equation to a definition of \lean{F} in terms of the appropriate construction on QPF, such that: (a) \lean{F} is known to be a QPF, and (b) the desired equality indeed holds.

As the name ``composition pipeline'' alludes, we are defining compositions of QPFs. Formally, \lean{MvQpf.Comp} has signature:
\begin{center}
    \lean{TypeFun n → Vec (TypeFun m) n → TypeFun m}
\end{center}
An $n$-ary type function \lean{F} is composed with an $n$-sized vector of $m$-ary type functions, resulting in an $m$-ary type function. The composition is essentially defined as
%LEANIGNORE
\begin{leancode}
    (Comp F ![G₁, ..., Gₙ]) ![α₁, ..., αₘ] 
                = F ![G₁ ![α₁, ..., αₘ], ..., Gₘ ![α₁, ..., αₘ]]
\end{leancode}
All arguments \lean{αᵢ} are broadcast to all functors \lean{Gⱼ}, meaning we don't have to worry about argument duplication or reordering.

Continuing with the example, we are going to define \lean{F} as \lean{Comp QpfTree.Shape ![G₁, G₂, G₃, G₄]}, for some functors \lean{G₁}, \lean{G₂}, \lean{G₃} and \lean{G₄}, satisfying the following equalities:
\begin{leancode}
    G₁ α β ρ = α                G₃ α β ρ = ρ
    G₂ α β ρ = β                G₄ α β ρ = (QpfList ρ)
\end{leancode}

At which point we are effectively in the same situation as before, wishing to define QPFs \lean{Gᵢ}
that satisfy certain equalities, betraying the recursive nature of the composition pipeline.

Before we continue, it should be clarified that the composition pipeline does not support any kind
of equation. For example, we cannot define a QPF that satisfies
\begin{badleancode}
    H α β = α → β 
\end{badleancode}
Because the arrow type constructor \lean{(⋅ → ⋅)} is \emph{not} functorial. 
At least, it is not functorial in both arguments. If, instead, we fix any value \lean{α} for the
first argument, we obtain \lean{(α → ⋅)}, which \emph{is} a QPF.
More generally, we say that in \lean{α → β}, α is a \emph{dead} variable, since it occurs as a non-functorial argument, while β is a \emph{live} variable.

By default, the composition pipeline assumes arguments are live, but we can explicitly mark some as dead
by giving a type ascription. Do note that the result is itself only functorial in the live arguments.
\begin{leancode}
    H (α : Type u) β = α → β 
\end{leancode}
That is to say, \lean{H} is not a QPF, but \lean{H a}, for arbitrary values \lean{a : α} is.




Continuing on, the composition pipeline does support these three kinds of functors:
\begin{itemize}
    \item \emph{Projections: } The right-hand-side of the equation is just a parameter, as in \lean{G₁ α β ρ = α}
    \item \emph{Constants: } The rhs does not mention the live parameters at all, e.g., \lean{G α β = Nat} or 
                \leanm{G (n : Nat) β = PFin2 n}.
    \item \emph{Compositions: } The rhs is an application of a QPF
\end{itemize}

This notably excludes dependent arrows \lean{(a : α) → \_}, anonymous functions \leanm{fun γ => _},
and applications of non-QPFs (such as non-dependent arrows (⋅ → ⋅) in its general form).



Projections are trivial to identify, and for constants we only have to verify that the target expression contains none of the live parameters. 
In all other cases, we try to find the largest expression \lean{G}, such that \lean{G} is a $k$-ary QPF, for some $k$, and we can write the target equation as
\lean{F ... = G e₁ ... eₖ}.
If no such \lean{G} exists, it means the original equation was not valid.


Then, we recursively call the composition pipeline, each time defining a new functor \lean{Hᵢ ... = eᵢ}, where \lean{Hᵢ} takes the same parameters as the original desired functor \lean{F}.

Finally, \lean{F} can simply be defined as the composition of \lean{G} with \lean{![H₁, ..., Hₘ]}.





































% We will start by giving example specifications, and showing how to represent each as a (co)fixpoint
% of a QPF manually.

% The \leanm{inductive} keyword encompasses three concepts: sums, products, and recursion.
% Let's forget about the latter for a second, and look at a non-recursive type
% \begin{leancode}
%     inductive Foo (α β γ : Type)
%     | bar : α → β → Foo α β γ
%     | qux : γ → Foo α β γ
% \end{leancode}
% We can replace the two arguments of the \leanm{bar} constructor by their product.
% \begin{leancode}
%     inductive Foo₂ (α β γ : Type)
%     | bar : (α × β) → Foo₂ α β γ
%     | qux : γ → Foo₂ α β γ
% \end{leancode}
% Having two constructor with a single argument just means taking their sum.
% \begin{leancode}
%     def Foo₃ (α β γ : Type) :=
%         (α × β) ⊕ β
% \end{leancode}




% \section{Polynomial Functors}

% Recall that we formalized polynomial functors, and their action on types, as  
% \begin{leancode}
%     structure MvPFunctor (n : Nat) :=
%         (A : Type u) (B : A → TypeVec.{u} n)

%     def MvPFunctor.Obj (P : MvPFunctor n) : TypeFun n
%         := fun α => Σ a : P.A, P.B a ⟹ α
% \end{leancode}

% That is, a polynomial functor $P$ is defined by some type $A$, where each element represents some 
% constructor of type $P(x_0, ..., x_{n-1})$, and a family of types $B_{a, i}$, with $a ∈ A$ and $i < n$, 
% where each element of $B_{a,i}$ is a label for an argument of type $x_i$ needed by constructor $a$.

% These can of course be any type, and this flexibility is useful in certain constructions,
% but ultimately the functor is defined by their \emph{cardinality}, i.e., how many inhabitants types 
% $A$ and $B_{a,i}$ have. Any additional structure of these types is not relevant for the behaviour of 
% the functor.

% Consequently, we can express any polynomial functor (up to isomorphism, at least) with the following
% shorthand.

% \begin{leancode}
%     def MvPFunctor.mk' {n : Nat} (ctors : List (Vec Nat n)) 
%         : MvPFunctor n
%     :=  let A := PFin2 ctors.length
%         let B := fun a i => PFin2 (ctors.get a.toFin i)
%         ⟨A, B⟩
% \end{leancode}

% Here \leanm{PFin2 n} is the type of natural numbers strictly less than $n$, which gives us an easy
% way to construct types of arbitrary (but finite) cardinality $n$.

% In effect, \leanm{ctors} is a matrix of fixed width $n$ (but arbitrary height). 
% Each row represents a constructor, the value in the $i$-th column tells us how many elements of the 
% $i$-th type argument are required in the constructor.

% \subsubsection*{Products \& Sums}
% For example, consider a (binary) product, which could be inductively defined as 
% \begin{leancode}
%     inductive Prod α β
%       | mk : α → U → Prod T U
% \end{leancode}

% We can define the corresponding functor, using our abbreviation as
% \begin{leancode}
%     def Prod.P : MvPFunctor 2 
%       := .mk' [
%         ![1, 1]
%       ]
% \end{leancode}
% At a glance, we see that \leanm{P ![α, β]} has a single constructor, and this constructor takes one
% argument of type \leanm{α} and one argument of \leanm{β}.
% This definition corresponds to the following
% \begin{leancode}
%     def Prod.P' : MvPFunctor 2 
%       :=  let A := PFin2 1
%           let B := fun _ => ![PFin2 1, PFin2 1]
%           ⟨A, B⟩
% \end{leancode}



% Another fundamental functor is the (binary) sum. Inductively, it looks like this
% \begin{leancode}
%     inductive Sum α β 
%       | inl : α → Sum α β 
%       | inl : β → Sum α β 
% \end{leancode}
% It has two constructors, so we use a matrix with two rows to define the polynomial functor.
% \begin{leancode}
%     def Sum.P : MvPFunctor
% \end{leancode}




% \subsection*{Universe polymorphism}

% As an aside, Lean does provide a standard \leanm{Fin n} type which encodes naturals less than $n$.
% However, \leanm{Fin n} has type \leanm{Type}, which forces the arguments to any functor \leanm{P} with
% \leanm{P.A = Fin n} to be in \leanm{Type 0} as well.

% To illustrate, here is a definition of the product functor, but now in terms of \leanm{Fin}
% \begin{leancode}
%     def Prod.Q : MvPFunctor 2 
%       :=  let A := Fin 1
%           let B := fun _ => ![Fin 1, Fin 1]
%           ⟨A, B⟩

%     -- `Nat` lives in `Type', so both functors are fine
%     #check  
% \end{leancode}
% First, consider the pair \leanm{(Nat, Nat)}. \leanm{Nat} lives in \leanm{Type}, so both functors are 
% fine
% \begin{leancode}
%     #check (P.Obj ![Nat, Nat] : Type)       -- ✓
%     #check (Q.Obj ![Nat, Nat] : Type)       -- ✓
% \end{leancode}
% Then, suppose some \leanm{X} that lives in a higher universe.
% \leanm{P} is able to adjust. 
% \begin{leancode}
%     variable (X : Type 1)
%     #check (P.Obj ![X, X] : Type 1)         -- ✓
% \end{leancode}
% However, \leanm{Q} complains that it expects \leanm{X} to live in \leanm{Type}.
% \begin{leancode}    
%     #check Q.Obj ![X, X]                    -- ×            
%     -- application type mismatch
%     --   Vec.append1 Vec.nil X
%     -- argument
%     --   X
%     -- has type
%     --   Type 1 : Type 2
%     -- but is expected to have type
%     --   Type : Type 1
% \end{leancode}

% Note that \leanm{P} still requires all arguments to live in the same universe, so
% \begin{leancode}
%     #check P.Obj ![X, Nat]                  -- ×
% \end{leancode}
% %     -- application type mismatch
% %     --   Vec.append1 (Vec.append1 Vec.nil ℕ) X
% %     -- argument
% %     --   X
% %     -- has type
% %     --   Type 1 : Type 2
% %     -- but is expected to have type
% %     --   Type : Type 1
% % \end{leancode}
% does not typecheck.



% \subsection*{Sum}
% For example, the type \leanm{Sum T U}, representing the disjoint union of types \leanm{T} and \leanm{U},
% is a polynomial functor. It could be specified as:
% \begin{leancode}
%     data Sum T U
%     | inl : T → Sum T U
%     | inr : U → Sum T U
% \end{leancode}
% For $A$, we take a simple enumeration of the constructors
% \begin{leancode}
%     inductive Sum.A
%     | inl : Sum.A
%     | inr : Sum.A
% \end{leancode}
% For $B$, we use the following
% \begin{leancode}
%     def Sum.B : Sum.A → TypeVec n
%     | .inl => ![Unit, Empty]
%     | .inr => ![Empty, Unit]
% \end{leancode}
% This means that an element of \leanm{Sum T U} made with constructor \leanm{inl} contains one inhabitant of \leanm{T} (\leanm{Unit} has exactly one inhabitant) and no inhabitants of \leanm{U} (as the name implies, \leanm{Empty} is not inhabited). Conversely, elements made with constructor \leanm{inr} contain one inhabitant of \leanm{U} and no inhabitant of \leanm{T}.


