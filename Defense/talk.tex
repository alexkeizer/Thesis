\batchmode % Do not display issues with package loading
\documentclass[t,12pt]{beamer}

% --------------------------------------------------------------------
% Load packages


%
%   PACKAGES   
%

\usepackage{url}
\usepackage{hyperref}
\usepackage{illcmolthesis}
\usepackage{amsmath}
\usepackage{amssymb}
\usepackage{unicode-math}
\usepackage{verbatim}   % for the comment environment
\usepackage{cleveref}
\usepackage{xcolor}
\usepackage{framed}

% 
%   Minted
% 

\usepackage[outputdir=./out]{minted}

% instruct minted to use our local theorem.py
\newmintinline[leanm]{lean4.py:Lean4Lexer -x}{fontsize=\small}
\newminted[leancode]{lean4.py:Lean4Lexer -x}{fontsize=\small}

% Lean code that does not typecheck/compile
\newminted[badleancode]{lean4.py:Lean4Lexer -x}{%
    fontsize=\small,%
    frame=leftline,%
    framesep=0mm,%
    rulecolor=red%
}

% Disable the red boxed minted will put around "syntax errors" 
\AtBeginEnvironment{minted}{%
  \renewcommand{\fcolorbox}[4][]{#4}}




%   
%   Fonts & coloring
% 

\usepackage[utf8]{inputenc}
\usepackage{newunicodechar}
\usepackage{fontspec}



% switch to a monospace font supporting more Unicode characters
\setmonofont{JetBrains Mono NL Light}
% \setmonofont{Droid Sans Mono}

\newfontfamily{\freemono}{FreeMono}
\newfontfamily{\droidmono}{Droid Sans Mono}
\newfontfamily{\jbmono}{JetBrains Mono NL Light}


% Colors
\definecolor{keywordcolor}{HTML}{008000}
\definecolor{operatorcolor}{HTML}{008000}

% Unicode glyphs

\newcommand\leanoperator[1]{%
\ifx\leanmode\undefined%
#1%
\else%
{\color{operatorcolor} #1}%
\fi%
}
\newcommand\leanmathoperator[1]{\ensuremath{\leanoperator{#1}}} 
\newunicodechar{→}{\leanmathoperator{\rightarrow}}
\newunicodechar{⟹}{\leanmathoperator{\Longrightarrow}}
\newunicodechar{⋅}{\leanmathoperator{\cdot}}

\newunicodechar{α}{{\droidmono α}}
\newunicodechar{β}{{\droidmono β}}
\newunicodechar{γ}{{\droidmono γ}}
\newunicodechar{₁}{\ensuremath{_\text{1}}}
\newunicodechar{₂}{\ensuremath{_\text{2}}}
\newunicodechar{ᵢ}{\ensuremath{_\text{i}}}
\newunicodechar{ⱼ}{\ensuremath{_\text{j}}}
\newunicodechar{ₖ}{\ensuremath{_\text{k}}}
\newunicodechar{ₙ}{\ensuremath{_\text{n}}}
\newunicodechar{ₘ}{\ensuremath{_\text{m}}}
\newunicodechar{₋}{\ensuremath{_\text{-}}}

% 
% \usepackage[GreekAndCoptic]{ucharclasses}
% \setTransitions{GreekAndCoptic}{}{}


% 
%   Spacing & Layout
% 

\usepackage{geometry}

\geometry{
    hmargin=10em
}
\frenchspacing
\setlength{\parskip}{6pt}
\setlength{\parindent}{0pt}



% 
%   Custom environments
% 
\definecolor{shadecolor}{HTML}{F8E0E0}

% definitions
\newenvironment{definition}[1][Definition:]{\begin{trivlist}                         
    \item[\hskip \labelsep {\bfseries #1}]}{\end{trivlist}}
    
% remark    
% \newenvironment{remark}{\begin{trivlist}                         
%   \item[\hskip \labelsep {\bfseries Remark:}]}{\end{trivlist}}



\newenvironment{remark}{%
% \def\FrameCommand{\colorbox{remarkcolor}}%
% \MakeFramed{\advance\hsize-\width \FrameRestore}%
\begin{framed}
\begin{trivlist}
    \item[\hskip \labelsep {\bfseries Remark:}]}%
{%
\end{trivlist}%
\end{framed}
    % \endMakeFramed%
}
    

% todo
\newenvironment{todo}{%
\definecolor{shadecolor}{HTML}{F8E0E0}%
\begin{shaded}%
\begin{trivlist}                         
    \item[\hskip \labelsep {\bfseries Todo:}]}{\end{trivlist}\end{shaded}}

% hidden code    
\newenvironment{leanhidden}{\expandafter\comment}{\expandafter\endcomment}



% 
%   Custom commands
% 

\newcommand\lean[1]{%
\ifx\leanmode\undefined%
\def\leanmode{1}%
\texttt{\small #1}%
\undef\leanmode%%
\else%
\texttt{#1}%
\fi%
}

\newcommand\keyword[1]{{\color{keywordcolor} \textbf{\lean{#1}}}}

\newcommand\inductive{{\keyword{inductive}}}
\newcommand\data{\keyword{data}}
\newcommand\codata{\keyword{codata}}
\newcommand\ldef{\keyword{def}}

\newcommand\Type{\leanm{Type}}
\newcommand\Typen[1]{\leanm{Type #1}}


\newcommand\etal{\emph{et al.}}

\usepackage{unicode-math}
\usepackage{fontspec}
\usepackage[outputdir=./out]{minted}
% instruct minted to use our local theorem.py
\newmintinline[lean]{lean4.py:Lean4Lexer -x}{bgcolor=white}
\newminted[leancode]{lean4.py:Lean4Lexer -x}{fontsize=\small}

\usepackage{booktabs}
\usepackage{graphicx}
\usepackage{xcolor}
\usepackage{tikz}
\usetikzlibrary{calc,trees,positioning,arrows,chains,shapes.geometric,%
    decorations.pathreplacing,decorations.pathmorphing,decorations.markings,shapes,%
    matrix,shapes.symbols,automata,cd,backgrounds, arrows.meta}
\usetikzlibrary{positioning}
\usepackage{color}
\usepackage{mathtools}

% \usepackage{listings}
% \lstset{tabsize=2,showspaces=false,showtabs=false,basicstyle=\ttfamily\mdseries\itshape\normalsize}

\usepackage{algpseudocode}
\usepackage{algorithm}
\usepackage[scr=boondoxo]{mathalpha}
\usepackage[normalem]{ulem}

\usepackage{qtree}

% --------------------------------------------------------------------
% Beamer version theme settings
\usetheme{Frankfurt}
% \usecolortheme{beaver}

% Remove navigation symbols
\setbeamertemplate{navigation symbols}{}

% Better default font
\usepackage{lmodern}
\usepackage[textfont={scriptsize,it}]{caption}
\setbeamerfont{caption}{size=\scriptsize}
\renewcommand*{\familydefault}{\sfdefault}

\usetikzlibrary{overlay-beamer-styles}

% switch to a monospace font supporting more Unicode characters
\setmonofont{FreeMono}

% --------------------------------------------------------------------

\def\liketitle#1{%
{\usebeamerfont{frametitle}\usebeamercolor[fg]{frametitle}%
\begin{flushleft}%
\vspace{-\baselineskip}% Cometic correction for space introduced by flushleft
#1\par
\end{flushleft}%
\vspace{-\baselineskip}% Cosmetic correction for space introduced by flushleft
}%
\vspace{0.75\baselineskip}%
}

% --------------------------------------------------------------------
\setbeameroption{hide notes}
%\setbeameroption{show notes}
%\setbeameroption{show notes on second screen}

% --------------------------------------------------------------------
\nonstopmode % Include issues with the slides
\beamertemplatenavigationsymbolsempty
\setbeamertemplate{navigation symbols}{}

% ====================================================================
% Header settings
\def\lecturename{Lecture}
\title[]{A (co)datatype package for Lean 4}
\subtitle{Based on Quotients of Polynomial Functors}
\date{}
\author{\textbf{Alex Keizer} \\ Supervised by: \and Jasmin Blanchette \and Benno v.d. Berg}
\subject{}


% ====================================================================
% Main part

% --------------------------------------------------------------------
\batchmode % Do not include issues with the package definitions
\begin{document}
\nonstopmode % Include issues with the slides



% ====================================================================
\section*{Introduction}

 % --------------------------------------------------------------------
 {
  \begin{frame}[plain]
      \maketitle
  \end{frame}
  \addtocounter{framenumber}{-1}% don't count the title slide.
 }

 % --------------------------------------------------------------------
 {
  \begin{frame}{Lean 4}
      \begin{itemize}
          \item Interactive Theorem Prover / Proof Assistant
          \item Dependently Typed
          \item Lean 4 is the latest release
      \end{itemize}

      \bigskip

    %   A previous formalization of QPFs exists, but in Lean 3
  \end{frame}
 }

% --------------------------------------------------------------------

% \begin{frame}[fragile]{Datatypes are inductive}
%     \bigskip
% \begin{leancode}
% inductive Nat
%   | zero : Nat
%   | succ : Nat → Nat
% \end{leancode}

% \bigskip

% That is, 
% \begin{itemize}
%     \item \lean{zero} is a \lean{Nat},
%     \item given any \lean{n : Nat}, \lean{succ n} is a \lean{Nat},
%     \item Nothing else is a \lean{Nat}
% \end{itemize}

% \medskip

% % Furthermore, we have \lean{Zero ≠ Succ n}
 
% \end{frame}

% --------------------------------------------------------------------


\begin{frame}[fragile]{Datatypes are inductive}
    \bigskip
\begin{leancode}
inductive List (α : Type)
  | nil  : List α
  | cons : α → List α → List α
\end{leancode}

\bigskip

\only<1>{
That is, 
\begin{itemize}
    \item \lean{nil} is a \lean{List α},
    \item If \lean{hd : α} and \lean{tl : List α}, then \\ \lean{cons hd tl} is a \lean{List α},
    \item Nothing else is a \lean{List α}
\end{itemize}
}

\only<2> {
Furthermore,
\begin{itemize}
    \item \lean{nil ≠ cons hd tl}
    \item Constructors are injective
\end{itemize}
}
\end{frame}



% --------------------------------------------------------------------


\begin{frame}[fragile]{Datatypes are inductive}
    \bigskip
\begin{leancode}
inductive List (α : Type)
  | nil  : List α
  | cons : α → List α → List α
\end{leancode}

\bigskip
Finally, the recursion/induction principle
\medskip

\begin{leancode}
List.rec : {α : Type} →
    {C : List α → Sort _} →
    C nil → 
    ((hd : α) → (tl : List α) 
        → C tl → C (cons hd tl)) → 
    (t : List α) →
    C t
\end{leancode}

\end{frame}
    
% --------------------------------------------------------------------


\begin{frame}[fragile]{Codatatypes are coinductive}
    \bigskip
\begin{leancode}
coinductive Stream (α : Type)
  | cons (head : α) (tail: Stream α) 
            : Stream α
\end{leancode}

     
    \end{frame}
    
% --------------------------------------------------------------------


\begin{frame}[fragile]{Codatatypes are coinductive}
    \bigskip
\begin{leancode}
coinductive Stream (α : Type)
  | cons (head : α) (tail: Stream α) 
            : Stream α
\end{leancode}

\bigskip
With its corecursor
\medskip

\begin{leancode}
Stream.corec {α β} : 
    (β → α × β) → β → Stream α 
\end{leancode}

Which satisfies
\begin{leancode}
head (Stream.corec f b) = fst (f b)
tail (Stream.corec f b) 
        = Stream.corec f (snd (f b))
\end{leancode}

     
    \end{frame}
    
% --------------------------------------------------------------------

\begin{frame}[fragile]{Lean has inductive}

    \medskip
    Lean already supports \lean{inductive}, but not \lean{coinductive} 
    declarations, based on a different system than I will explain today.

    \bigskip

    My goal: implement \lean{data} and \lean{codata} declarations 
    that compile down to QPF-constructions.
    
\end{frame}

% --------------------------------------------------------------------

\section*{QPFs}

\begin{frame}{What's in a dataype}
    A \lean{List α}, e.g., \lean{[1,2,3]} is a ``tree'':
    \begin{center}
        \Tree [.\lean{cons 1} [.\lean{cons 2} [.\lean{cons 3} \lean{nil} ] ]]
    \end{center}

    \bigskip
    In general:

    \begin{columns}
        \begin{column}{0.49\linewidth}\centering
            \Tree [.\lean{cons α} \lean{List α} ]    
        \end{column}%
        \begin{column}{0.49\linewidth}\centering
            \lean{nil}
        \end{column}
    \end{columns}
    
\end{frame}

\begin{frame}
    \begin{definition}
        A polynomial functor is a functor that is isomorphic to
        \[
            P(α) = \sum_{a ∈ A} B(a) → α
        \]
        For some set $A$ and $A$-indexed family of sets $B(a)_{a ∈ A}$
    \end{definition}

    \bigskip

    An element of $P(α)$ is a (dependent) pair $⟨a, f⟩$ where $a ∈ A$ is the ``shape'' and 
    $f : B(a) → α$ is the ``data''.

    \bigskip
   
\end{frame}

% \begin{frame}[fragile]
%     \bigskip
%     \begin{leancode}
% structure PFunctor :=
%     A : Type u
%     B : A → Type u

% /-- Applying `P` to an object of `Type` -/
% def Obj (P : PFunctor) (α : Type u)
%   := Σ x : P.A, P.B x → α

% /-- Applying `P` to a morphism of `Type` -/
% def map (P : PFunctor) (f : α → β) 
%                 : P.Obj α → P.Obj β :=
%   λ ⟨a, g⟩ => ⟨a, f ∘ g⟩
%     \end{leancode}    
% \end{frame}

\begin{frame}
    \begin{definition}
        The $W$-type of a polynomial functor $P$ is its initial algebra
    \end{definition}

    \medskip

    An element of $W_P$ is a (dependent) pair $⟨a, f⟩$ where $a ∈ A$ and $f : B(a) → W_P$

    \bigskip

    \only<2>{
        For example, let $P_α$ be the polynomial functor defined by
        \begin{align*}
            A           &:= \{nil\} ∪ (\{cons\} × α) \\
            B(nil)      &:= \emptyset \\
            B(cons, n)  &:= {*}
        \end{align*}
        Then, the corresponding $W$-type models \lean{List α}
    }
\end{frame}

\begin{frame}[fragile]{Beyond inductive}
    A \lean{MultiSet α} is quotient of \lean{List α} (that identifies lists up-to permutation).

    \bigskip

    Just polynomial functors cannot encode \lean{FinSet α}, hence 
    Quotients of Polynomial Functors.
\end{frame}

\begin{frame}[fragile]{Beyond inductive}
    A \lean{MultiSet α} is quotient of \lean{List α} (that identifies lists up-to permutation).

    \bigskip

    Just polynomial functors cannot encode \lean{FinSet α}, hence 
    Quotients of Polynomial Functors.

    \bigskip

    QPFs are closed under a bunch of constructions that allow us to arbitrarily nest inductive, 
    coinductive and quotient types.

    \bigskip

    For example, the following defines infinite trees with unordered children.
    \begin{leancode}
codata InfTree (α : Type)
  | α → MultiSet (InfTree α) → InfTree α
    \end{leancode}
\end{frame}



\section*{Practicalities}

\begin{frame}{My Project}
    The theory of QPFs has been worked out, and the constructions formalized in lean 3, 
    by Jeremy Avigad, Mario Carneiro and Simon Hudon.

    \bigskip

    But, users don't want to manually define their polynomial functor, 
    they want to use nice \lean{inductive} syntax.

    \bigskip

    A big change from Lean 3 to Lean 4 is that Lean 4 is mostly written in Lean itself, 
    so we can look at how Lean parses \lean{inductive} declarations, and reuse a lot of that code
    to parse our new \lean{data} and \lean{codata} declarations.

\end{frame}

\begin{frame}{My Project}

    \begin{itemize}
        \alt<2->{\item[$\checkmark$]}{\item} Port existing work to Lean 4
        \alt<3->{\item[$\checkmark$]}{\item} Implement a \lean{data} macro that can parse nice syntax
        \item Generate the QPF
        \item Generate auxiliary constructions (e.g., induction principle)
        \alt<4->{\item[(?)]}{\item} Support nested datatypes, through composition
        \alt<4->{\item[(?)]}{\item} Now do the same for \lean{codata}
        \alt<4->{\item[(?)]}{\item} Quotients
    \end{itemize}
\end{frame}


\end{document}

