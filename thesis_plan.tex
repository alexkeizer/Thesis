\documentclass[a4paper]{article}

\frenchspacing
\setlength{\parskip}{6pt}
\setlength{\parindent}{0pt}

\title{Thesis Plan}
\author{Alex Keizer}
\date{}

\begin{document}
\maketitle

Supervised by Jasmin Blanchette (from VU's CS department) and Benno van den Berg I plan to work on a 
datatype package for Lean 4, based on quotients of polynomial functors (QPFs).

Prior literature has explored the theory of QPFs, and provided a formalization in Lean 3. \cite{qpf}
The goal of this project will be to provide a more high-level, user-friendly way to define (co)datatypes.
A similar package was also developed for Isabelle/HOL, using the slightly different construct of BNFs. \cite{isabelle}

Firstly, we will be working on porting the existing formalisation to Lean 4.

Then, we plan to write a parser, enabling users to declare datatypes using intuitive recursive equations,
rather than manually defining the underlying QPFs.

Using this parser, I will explore example definitions of simple datatypes, such as lists and Rose trees.

The final part will be to reduce user-friction.
For example, by automatically synthesizing high-level induction principles, or to automatically recognize when
a polymorphic datatype constitutes a QPF itself and to register it as such.


The primary contribution of the project will be the implementation of Lean code, 
rather than development of theory. 
As such, most time will be spent in the research/implementation phase, 
with roughly the final six weeks dedicated to writing up the thesis.

\begin{thebibliography}{10}
    \bibitem{qpf} \url{https://drops.dagstuhl.de/opus/volltexte/2019/11061/pdf/LIPIcs-ITP-2019-6.pdf}
    \bibitem{isabelle} https://isabelle.in.tum.de/dist/doc/datatypes.pdf    
\end{thebibliography}

\end{document}